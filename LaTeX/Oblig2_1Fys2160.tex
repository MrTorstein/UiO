\documentclass[11pt, A4paper, norsk]{article}
\usepackage[utf8]{inputenc}
\usepackage[T1]{fontenc}
\usepackage{babel}
\usepackage{amsmath}
\usepackage{amsfonts}
\usepackage{amsthm}
\usepackage{amssymb}
\usepackage[colorlinks]{hyperref}
\usepackage{listings}
\usepackage{color}
\usepackage{hyperref}
\usepackage{graphicx}
\usepackage{cite}
\usepackage{textcomp}
\usepackage{float}

\definecolor{dkgreen}{rgb}{0, 0.6, 0}
\definecolor{gray}{rgb}{0.5, 0.5, 0.5}
\definecolor{daynineyellow}{rgb}{1.0, 0.655, 0.102}
\definecolor{url}{rgb}{0.1, 0.1, 0.4}

\lstset{frame=tb,
	language=Python,
	aboveskip=3mm,
	belowskip=3mm,
	showstringspaces=false,
	columns=flexible,
	basicstyle={\small\ttfamily},
	numbers=none,
	numberstyle=\tiny\color{gray},
	keywordstyle=\color{blue},
	commentstyle=\color{daynineyellow},
	stringstyle=\color{dkgreen},
	breaklines=true,
	breakatwhitespace=true,
	tabsize=3
}

\lstset{inputpath="C:/Users/Torstein/Documents/UiO/Fys2160/Python programmer"}
\graphicspath{{C:/Users/Torstein/Documents/UiO/Fys2160/"Python programmer"/}}
\hypersetup{colorlinks, urlcolor=url}

\author{Torstein Solheim Ølberg}
\title{Svar på Oblig nr. 2 i Fys2160}



%\lstinputlisting{Filnavn! type kodefil}
%\includegraphics[width=12.6cm, height=8cm]{Filnavn! type png}



\begin{document}
\maketitle
	\begin{center}
\Large \textbf{Oppgaver}
	\end{center}









		\paragraph{1.}
			\subparagraph{1}
				\begin{gather*}
g(L) = \frac{a L}{L_0} \left( 1 - \left( \frac{L_0}{L} \right)^3 \right) \\
f(T, L) = \frac{a T}{L_0} \left( 1 + 2 \left( \frac{L_0}{L} \right)^3 \right) \\
\left( \frac{\partial f}{\partial T} \right)_L = \left( \frac{\partial g}{\partial L} \right)_T \\
\left( \frac{\partial f}{\partial T} \right)_L = \frac{a}{L_0} \left( 1 + 2 \left( \frac{L_0}{L} \right)^3 \right) \\
\left( \frac{\partial g}{\partial L} \right)_T = \frac{a L}{L_0} \left( 1 - \left( \frac{L_0}{L} \right)^3 \right) = \frac{a}{L_0} \left( 1 - \left( \frac{L_0}{L} \right)^3 \right) + \frac{a}{L_0} \left( 3 \left( \frac{L_0}{L} \right)^3 \right) \\
\left( \frac{\partial g}{\partial L} \right)_T = \frac{a}{L_0} \left( 1 - \left( \frac{L_0}{L} \right)^3 \right) + \frac{a}{L_0} \left( 3 \left( \frac{L_0}{L} \right)^3 \right) = \frac{a}{L_0} \left( 1 + 2 \left( \frac{L_0}{L} \right)^3 \right) \\
\text{Ser at de to uttrykene ble like og Maxwells relasjon stemmer da,} \\
\text{altså er $P$ en veldefinert funksjon}
				\end{gather*}









			\subparagraph{2)}
				\begin{gather*}
f(T, L) = \frac{a T}{L_0} \left( 1 + 2 \left( \frac{L_0}{L} \right)^3 \right) \\
g'(L) = \frac{a}{L_0} \left( 1 + 2 \left( \frac{L_0}{L} \right)^3 \right) \Rightarrow T g'(L) = \frac{a T}{L_0} \left( 1 + 2 \left( \frac{L_0}{L} \right)^3 \right) \\
\text{De to uttrykene er like, som var det vi skulle vise}
				\end{gather*}









			\subparagraph{3)}
				\begin{gather*}
P = \int g(L) dT = \int \frac{a L}{L_0} \left( 1 - \left( \frac{L_0}{L} \right)^3 \right) dT \\
P = \frac{a L T}{L_0} \left( 1 - \left( \frac{L_0}{L} \right)^3 \right) + C \\
P = \int f(T, L) dL = \int \frac{a T}{L_0} \left( 1 + 2 \left( \frac{L_0}{L} \right)^3 \right) dL \\
P = \frac{a T L}{L_0} \left( 1 - \left( \frac{L_0}{L} \right)^3 \right) + C \\
\text{De to uttrykene for $P$ er like så lenge $C$ er en konstant}
				\end{gather*}

			









			\subparagraph{4)}
				\begin{flushleft}
Siden trykket endrer seg jo mer du strekker strikken, men temperaturne alltid holdes likt, så er Helmholtz riktig type gratis energi.
				\end{flushleft}
				\begin{gather*}
F = U - TS \\
dU = T dS + P dL \\
F = T S + P (L - L_0) - T S = P (L - L_0)
				\end{gather*}









			\subparagraph{5)}
				\begin{gather*}
\left( \frac{\partial P}{\partial L} \right)_T
				\end{gather*}










			\subparagraph{6)}
				\begin{flushleft}
a
				\end{flushleft}









			\subparagraph{7)}
				\begin{flushleft}
a
				\end{flushleft}












			\subparagraph{8)}
				\begin{flushleft}
a
				\end{flushleft}











		\paragraph{2.}
			\subparagraph{1)}
				\begin{flushleft}
Varmekapasitet ved kostant volum er definert som
$$C_V = \left( \frac{\partial U}{\partial T} \right)_V$$
og varmekapasiteten ved konstant trykk er definert som
$$C_P = \left( \frac{\partial U}{\partial T} \right)_P + P \left( \frac{\partial V}{\partial T} \right)_P$$
				\end{flushleft}
				\begin{gather*}
C_P = \left( \frac{\partial U}{\partial T} \right)_P + P \left( \frac{\partial V}{\partial T} \right)_P \\
C_P = \left( \frac{\partial U}{\partial T} \right)_P + P \left( \frac{dV}{dT} - \left( \frac{\partial V}{\partial P} \right)_T \frac{dP}{dT} \right)
				\end{gather*}








			\subparagraph{2)}
				\begin{flushleft}
For ideel gass gjelder ideel gass lov
$$PV = NkT$$
som betyr at
$$C_P = C_V + T \left( \frac{\partial P}{\partial T} \right)_V \left( \frac{\partial V}{\partial T} \right)_P = C_V + T \frac{Nk}{V} \frac{Nk}{P} = C_V + \frac{N^2k^2T}{PV}$$
$$C_P = C_V + Nk$$
				\end{flushleft}










			\subparagraph{3)}
				\begin{gather*}
C_P = C_V + T \left( \frac{\partial P}{\partial T} \right)_V \left( \frac{\partial V}{\partial T} \right)_P \\
C_P - C_V = T \left( \frac{\partial P}{\partial T} \right)_V \left( \frac{\partial V}{\partial T} \right)_P \\
\left( \frac{\partial V}{\partial P} \right)_T \left( \frac{\partial P}{\partial T} \right)_V = - \left( \frac{\partial V}{\partial T} \right)_P \\
\left( \frac{\partial P}{\partial T} \right)_V = - \left( \frac{\partial V}{\partial T} \right)_P / \left( \frac{\partial V}{\partial P} \right)_T \\
C_P - C_V = T \left( - \left( \frac{\partial V}{\partial T} \right)_P / \left( \frac{\partial V}{\partial P} \right)_T \right) \left( \frac{\partial V}{\partial T} \right)_P \\
C_P - C_V = - T \left( \frac{\partial V}{\partial T} \right)^2_P / \left( \frac{\partial V}{\partial P} \right)_T \\
\kappa_T = - \frac{1}{V} \left( \frac{\partial V}{\partial P} \right)_T \Rightarrow \left( \frac{\partial V}{\partial P} \right)_T = - V \kappa_T \\
\beta = \frac{1}{V} \left( \frac{\partial V}{\partial T} \right)_P \Rightarrow \left( \frac{\partial V}{\partial T} \right)_P = V \beta \\
C_P - C_V = \frac{- T V^2 \beta^2}{- V \kappa_T} = \frac{T V \beta^2}{\kappa_T}
				\end{gather*}








			\subparagraph{4)}
				\begin{flushleft}
For ideel gass gjelder ideel gass lov
$$PV = NkT \vee V = \frac{NkT}{P}$$ 
som betyr at
$$\kappa_T = - \frac{1}{V} \left( \frac{\partial V}{\partial P} \right)_T = - \frac{P}{NkT} \left( \frac{\partial}{\partial P} \frac{NkT}{P} \right)_T = \frac{P}{NkT} \frac{NkT}{2 P^2} = \frac{1}{2 P}$$
Da er
$$\beta = \frac{1}{V} \left( \frac{\partial V}{\partial T} \right)_P = \frac{P}{NkT} \left( \frac{\partial}{\partial T} \frac{NkT}{P} \right)_P = \frac{P}{NkT} \frac{Nk}{P} = \frac{1}{T}$$
				\end{flushleft}
\end{document}