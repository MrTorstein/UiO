\documentclass[8pt, A4paper, norsk]{extarticle}
\usepackage[utf8]{inputenc}
\usepackage[T1]{fontenc}
\usepackage{babel}
\usepackage{amsmath}
\usepackage{amsfonts}
\usepackage{amsthm}
\usepackage{amssymb}
\usepackage[colorlinks]{hyperref}
\usepackage{listings}
\usepackage{color}
\usepackage{hyperref}
\usepackage{graphicx}
\usepackage{cite}
\usepackage{textcomp}
\usepackage{float}
\usepackage{multicol}
\usepackage[margin=11pt]{geometry}

\definecolor{dkgreen}{rgb}{0,0.6,0}
\definecolor{gray}{rgb}{0.5,0.5,0.5}
\definecolor{daynineyellow}{rgb}{1.0,0.655,0.102}
\definecolor{url}{rgb}{0.1,0.1,0.4}

\lstset{frame=tb,
	language=Python,
	aboveskip=3mm,
	belowskip=3mm,
	showstringspaces=false,
	columns=flexible,
	basicstyle={\small\ttfamily},
	numbers=none,
	numberstyle=\tiny\color{gray},
	keywordstyle=\color{blue},
	commentstyle=\color{daynineyellow},
	stringstyle=\color{dkgreen},
	breaklines=true,
	breakatwhitespace=true,
	tabsize=3
}

\lstset{inputpath="C:/Users/Torstein/Documents/UiO/<Faget>/Python programmer"}
\graphicspath{{C:/Users/Torstein/Documents/UiO/<Faget>/"Python programmer"/}}
\hypersetup{colorlinks, urlcolor=url}



\begin{document}
	\begin{center}
\Large \textbf{Jukseark}
	\end{center}
	\begin{multicols*}{3}
		\begin{gather*}
\text{\textbf{Tema 1:}} \\
a
		\end{gather*}





		\begin{gather*}
\text{\textbf{Tema 2:}} \\
a
		\end{gather*}





		\begin{gather*}
\text{\textbf{Tema 3:}} \\
a
		\end{gather*}






		\begin{gather*}
\text{\textbf{Tema 4:}} \\
\text{Underoverskrift} \\
a \\
\text{Underoverskrift} \\
a \\
\text{Underoverskrift} \\
a \\
\text{Underoverskrift} \\
a
		\end{gather*}






		\begin{gather*}
\text{\textbf{Tema 5:}} \\
\text{Potensial} \\
a
		\end{gather*}






		\begin{gather*}
\text{\textbf{Tema 6:}} \\
a
		\end{gather*}






		\begin{gather*}
\text{\textbf{Tema 7:}} \\
a
		\end{gather*}







		\begin{gather*}
\text{\textbf{Tema 8:}} \\
a
		\end{gather*}







		\begin{gather*}
\text{\textbf{Andre Formler:}} \\
a
		\end{gather*}








\centering \textbf{Tema 9:} \\
a

\vspace{2mm}







\centering \textbf{Tema 10:} \\
a \\
a

\vspace{2mm}









\centering \textbf{Tema 11:} \\
a

\vspace{2mm}







\centering \textbf{Tema 12:} \\
a \\
a

\vspace{2mm}






\centering \textbf{Tema 13:} \\
a

\vspace{2mm}







\centering \textbf{Tema 14:} \\
a

\vspace{2mm}








\centering \textbf{Tema 15:} \\
a
		\begin{align*}
&a
		\end{align*}
		\begin{flushleft}
a
		\end{flushleft}
\end{multicols*}
\end{document}