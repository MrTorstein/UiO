\documentclass[11pt, A4paper,norsk]{article}
\usepackage[utf8]{inputenc}
\usepackage[T1]{fontenc}
\usepackage{babel}
\usepackage{amsmath}
\usepackage{amsfonts}
\usepackage{amsthm}
\usepackage{amssymb}
\usepackage[colorlinks]{hyperref}
\usepackage{listings}
\usepackage{color}
\usepackage{hyperref}
\usepackage{graphicx}
\usepackage{cite}
\usepackage{textcomp}
\usepackage{float}

\definecolor{dkgreen}{rgb}{0,0.6,0}
\definecolor{gray}{rgb}{0.5,0.5,0.5}
\definecolor{daynineyellow}{rgb}{1.0,0.655,0.102}
\definecolor{url}{rgb}{0.1,0.1,0.4}

\lstset{frame=tb,
	language=Python,
	aboveskip=3mm,
	belowskip=3mm,
	showstringspaces=false,
	columns=flexible,
	basicstyle={\small\ttfamily},
	numbers=none,
	numberstyle=\tiny\color{gray},
	keywordstyle=\color{blue},
	commentstyle=\color{daynineyellow},
	stringstyle=\color{dkgreen},
	breaklines=true,
	breakatwhitespace=true,
	tabsize=3
}

\lstset{inputpath="C:/Users/Torstein/Documents/UiO/Fys3140/Python programmer"}
\graphicspath{{C:/Users/Torstein/Documents/UiO/Fys3140/"Python programmer"/}}
\hypersetup{colorlinks, urlcolor=url}

\author{Torstein Solheim Ølberg}
\title{Svar på Oblig nr. 5 i Fys3140}



%\lstinputlisting{Filnavn! type kodefil}
%\includegraphics[width=12.6cm,height=8cm]{Filnavn! type png}



\begin{document}
\maketitle
	\begin{center}
\Large \textbf{Oppgaver}
	\end{center}









		\paragraph{6)}
			\begin{flushleft}
Jeg sjekker resultatene mine på datamaskin ved å skrive dem inn på siden \url{https://www.wolframalpha.com} på måten '' Residue of $f(z)$ at $z = z0$ ''
			\end{flushleft}
			\subparagraph{15.}
				\begin{gather*}
f(z) = \frac{1}{(1 - 2z)(5z - 4)} \\
\text{Finner residuen i punktet $z = \frac{1}{2}$, dette er en første ordens pole.} \\
R\left( \frac{1}{2} \right) = \lim_{z \rightarrow \frac{1}{2}} \left( z - \frac{1}{2} \right) f(z) \\
\lim_{z \rightarrow \frac{1}{2}} \left( z - \frac{1}{2} \right)  \frac{1}{(1 - 2z)(5z - 4)} = \lim_{z \rightarrow \frac{1}{2}} \frac{\left( z - \frac{1}{2} \right)}{(1 - 2z)(5z - 4)} \\
\lim_{z \rightarrow \frac{1}{2}} \frac{\left( z - \frac{1}{2} \right)}{-10z^2 + 8z + 5z - 4} \xrightarrow{\text{L'Hôpitals}} \lim_{z \rightarrow \frac{1}{2}} \frac{1}{-20z + 13} = \frac{1}{13 - \frac{20}{2}} \\
R\left( \frac{1}{2} \right) = \frac{1}{13 - 10} = \frac{1}{3} \\
\text{Finner for $z = \frac{4}{5}$. Dette er også en førsteordens pole.} \\
R\left( \frac{4}{5} \right) = \lim_{z \rightarrow \frac{4}{5}} \frac{ z - \frac{4}{5}}{(1 - 2z)(5z - 4)} = \lim_{z \rightarrow \frac{4}{5}} \frac{1}{13 - 20z} = \frac{1}{13 - 16} = - \frac{1}{3}
				\end{gather*}
				\begin{flushleft}
I følge datamaskinen stemmer begge disse svarene.
				\end{flushleft}








			\subparagraph{15'.}
				\begin{gather*}
\oint_C f(z) dz = 2 \pi i \cdot \sum_{n = 0}^{N}(R_n( z_n )) \\
\text{$N$ blir i dette tilfellet $1$ fordi begge singularitetene ligger inni $C$} \\
\oint_C \frac{1}{(1 - 2z)(5z - 4)} dz = 2 \pi i \cdot \left( \frac{1}{3} - \frac{1}{3} \right) = 0
				\end{gather*}









			\subparagraph{19.}
				\begin{gather*}
f(z) = \frac{\sin^2(z)}{2z - \pi} \\
\text{Finner residue i $z = \frac{\pi}{2}$ som vil være en første ordens pole} \\
R\left( \frac{\pi}{2} \right) = \lim_{z \rightarrow \frac{\pi}{2}} \frac{\left( z - \frac{\pi}{2} \right) \sin^2(z)}{2z - \pi} \xrightarrow{\text{L'Hôpitals}} \lim_{z \rightarrow \frac{\pi}{2}} \frac{\sin^2(z) + z\sin(2z) - \frac{\pi}{2} \sin(2z)}{2} \\
R\left( \frac{\pi}{2} \right) = \frac{1}{2}
				\end{gather*}
				\begin{flushleft}
Dette stemmer også i følge datamaskin.
				\end{flushleft}









			\subparagraph{28.}
				\begin{gather*}
f(z) = \frac{z + 2}{(z^2 + 9)(z^2 + 1)} \\
\text{Finner residue i $z = 3i$ som vil være en første ordens pole} \\
R( 3i ) = \lim_{z \rightarrow 3i} \frac{( z - 3i ) ( z + 2 )}{( z^2 + 9 )( z^2 + 1 )} \\
R( 3i ) = \lim_{z \rightarrow 3i} \frac{z^2 + z(2 - 3i) - 6i}{z^4 + 10z^2 + 9} \xrightarrow{\text{L'Hôpitals}} \lim_{z \rightarrow 3i} \frac{2z + 2 - 3i}{4z^3 + 20z} \\
R( 3i ) = \frac{6i - 3i + 2}{4 \cdot - 27i + 20 \cdot 3i} = \frac{2 + 3i}{60i - 108i} = \frac{2 + 3i}{- 48i} = \frac{i}{24} - \frac{1}{16}
				\end{gather*}
				\begin{flushleft}
Stemmer i følge datamaskin
				\end{flushleft}









			\subparagraph{28'.}
				\begin{gather*}
\text{Bare residuen ved punktet $z = i$ er innafor sirkelen $C$, så må regne ut denne} \\
R( i ) = \lim_{z \rightarrow i} \frac{2z + 2 - i}{4z^3 + 20z} = \frac{2i - i + 2}{20i - 4i} = \frac{2 + i}{16i} \\
\oint_C \frac{z + 2}{(z^2 + 9)(z^2 + 1)} dz = 2 \pi i \cdot R(i) = 2 \pi i \frac{2 + i}{16i} = \frac{2\pi + \pi i}{8}
				\end{gather*}









		\paragraph{7)}
			\subparagraph{6.}
				\begin{gather*}
I = \int_{0}^{\pi} \frac{d\theta}{(2 + \cos(\theta))^2} \xrightarrow{symmetri} \frac{1}{2} \int_{0}^{2 \pi} \frac{d\theta}{(2 + \cos(\theta))^2} \\
\text{Løser integralet ved å endre variabel til $z = e^{i\theta}$ og integrer rundt $|z| = 1$} \\
d\theta = \frac{dz}{iz} \text{ og} \cos(\theta) = \frac{z + \frac{1}{z}}{2} \\
I = \frac{1}{2} \oint_C \frac{\frac{1}{iz}}{\left( 2 + \frac{z + \frac{1}{z}}{2} \right)^2} = - i \oint_C \frac{2}{z \left( 4 + z + \frac{1}{z} \right)^2} = - i \oint_C \frac{2}{( z^3 + 4z^2 + z )^2} \\
I = - i \oint_C \frac{2}{( z( z^2 + 4z + 1 ) )^2} = - i \oint_C \frac{2}{z^2( ( z + 2 + \sqrt{3} )( z + 2 - \sqrt{3} ) )^2} \\
I = - i \oint_C \frac{2}{z^2( z + 2 + \sqrt{3} )^2 ( z + 2 - \sqrt{3} )^2} \\
\text{Finner residuene i punktene $z = 0$ og $z = \sqrt{3} - 2$, som begger er andre ordens poler} \\
R(0) = \lim_{z \rightarrow 0}\frac{d}{dz}\left( z^2 f(z) \right) = \lim_{z \rightarrow 0}\frac{d}{dz}\left( \frac{2}{( z + 2 + \sqrt{3} )^2 ( z + 2 - \sqrt{3} )^2} \right) \\
R(0) = \lim_{z \rightarrow 0} \frac{d}{dz} \left( \frac{2}{( z^2 + 4z + 1 )^2} \right) = \lim_{z \rightarrow 0} \left( \frac{-4( z^2 + 4z + 1 ) (2z + 4 )}{( z^2 + 4z + 1 )^4} \right) \\
R(0) =  \lim_{z \rightarrow 0} \frac{- 4( 2z + 4 )}{( z^2 + 4z + 1 )^3} = - \frac{16}{1^4} = -16 \\
R\left( \sqrt{3} - 2 \right) = \lim_{z \rightarrow \sqrt{3} - 2} \frac{d}{dz}\left( ( z + 2 - \sqrt{3} )^2 f(z) \right) = \lim_{z \rightarrow \sqrt{3} - 2} \frac{d}{dz}\left( \frac{2}{z^2 ( z + 2 + \sqrt{3} )^2} \right) \\
R\left( \sqrt{3} - 2 \right) = \lim_{z \rightarrow \sqrt{3} - 2} \frac{-4( 2z + 2 + \sqrt{3} )}{z^3 ( z + 2 + \sqrt{3} )^3} \\
R\left( \sqrt{3} - 2 \right) = \frac{-4( 3\sqrt{3} - 2 )}{( \sqrt{3} - 2 )^3 ( 2\sqrt{3} )^3} \\
R\left( \sqrt{3} - 2 \right) = \frac{8 - 12\sqrt{3}}{( \sqrt{3} - 2 )^3 ( 2\sqrt{3} )^3} \approx 15.99 \\
\text{Residu teoremet sider da at:} \\
I = - i \cdot 2 \pi i \cdot (-16 + 15.99) = - 0.02 \pi
				\end{gather*}









			\subparagraph{8.}
				\begin{gather*}
I = \int_{0}^{\pi} \frac{\sin^2(\theta) d\theta}{13 - 12\cos(\theta)} = \frac{1}{2} \int_{0}^{2\pi} \frac{\sin^2(\theta) d\theta}{13 - 12\cos(\theta)} \\
d\theta = \frac{dz}{zi} \text{, } \sin(\theta) = \frac{z - \frac{1}{z}}{2i} \text{ og} \cos(\theta) = \frac{z + \frac{1}{z}}{2} \\
I = \oint_{C} \frac{\frac{-(z - \frac{1}{z})^2}{4} \frac{dz}{zi}}{13 - 12\left(\frac{z + \frac{1}{z}}{2}\right)} = \frac{i}{24} \oint_{C} \frac{(z - \frac{1}{z})^2 dz}{- z^2 + \frac{13}{6}z + 1} = \frac{i}{24} \oint_{C} \frac{(z - \frac{1}{z})^2 dz}{-(z - \frac{13}{12} + \frac{\sqrt{313}}{12})(z - \frac{13}{12} - \frac{\sqrt{313}}{12})} \\
\text{Finner residu for $z = \frac{13 - \sqrt{313}}{12}$ som er en enkel pole} \\
R\left(\frac{13 - \sqrt{313}}{12}\right) = \lim_{z \rightarrow \frac{13 - \sqrt{313}}{12}}\left(z - \frac{13 - \sqrt{313}}{12}\right) f(z) = \lim_{z \rightarrow \frac{13 - \sqrt{313}}{12}} \frac{(z - \frac{1}{z})^2}{- z + \frac{13 + \sqrt{313}}{12}} \approx 1.5921 \\
I = 2 \pi i \frac{i}{24} (1.5921) = - \frac{1.5921 \pi}{12}
				\end{gather*}












			\subparagraph{10.}
				\begin{gather*}
I = \int_{- \infty}^{\infty} \frac{1}{x^2 + 4x + 5} dx = \int_{- \infty}^{\infty} \frac{1}{(x + 2 + i)(x + 2 - i)} dx \\
J = \oint_{C} \frac{1}{(z + 2 + i)(z + 2 - i)} dz \\
R(- 2 - i) = \lim_{z \rightarrow - 2 - i} (z + 2 + i) \frac{1}{(z + 2 + i)(z + 2 - i)} = \lim_{z \rightarrow - 2 - i} \frac{1}{(z + 2 - i)} \\
R(- 2 - i) = \frac{1}{-2i} = \frac{i}{2} \\
J = 2 \pi i \frac{i}{2} = - \pi \\
J = \lim_{r \rightarrow \infty} I + \int_{0}^{\pi} \frac{1}{r^2e^{2i\theta} + 4re^{i\theta} + 5} d\theta \Rightarrow I = J = -\pi
				\end{gather*}












			\subparagraph{12.}
				\begin{gather*}
I = \int_{0}^{\infty} \frac{x^2}{x^4 + 16} dx = \int_{0}^{\infty} \frac{x^2}{(x + 2\sqrt[4]{-1})(x - 2\sqrt[4]{-1})(x + 2\sqrt[4]{-1}^3)(x - 2\sqrt[4]{-1}^3)} dx \\
J = \oint_{C} \frac{z^2}{(z + 2\sqrt[4]{-1})(z - 2\sqrt[4]{-1})(z + 2\sqrt[4]{-1}^3)(z - 2\sqrt[4]{-1}^3)} dx \\
R(2\sqrt[4]{-1}) = \lim_{z \rightarrow 2\sqrt[4]{-1}} \frac{z^2}{(z + 2\sqrt[4]{-1})(z + 2\sqrt[4]{-1}^3)(z - 2\sqrt[4]{-1}^3)} \\
R(2\sqrt[4]{-1}) = \frac{4i}{(4\sqrt[4]{-1})(2(\sqrt[4]{-1} + \sqrt[4]{-1}^3))(2(\sqrt[4]{-1} - \sqrt[4]{-1}^3))} \\
R = \frac{4i}{4\sqrt[4]{-1} \cdot 2 \cdot 1.4142i \cdot 2 \cdot 1.4142} = \frac{1}{\frac{\sqrt{2} + \sqrt{2}i}{2} \cdot 8} = \frac{1}{4(\sqrt{2} + \sqrt{2}i)}
				\end{gather*}
\end{document}