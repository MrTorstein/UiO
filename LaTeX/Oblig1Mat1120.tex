\documentclass[11pt, A4paper,norsk]{article}
\usepackage[utf8]{inputenc}
\usepackage[T1]{fontenc}
\usepackage{babel}
\usepackage{amsmath}
\usepackage{amsfonts}
\usepackage{amsthm}
\usepackage[colorlinks]{hyperref}
\usepackage{listings}
\usepackage{color}
\usepackage{hyperref}
\usepackage{graphicx}
\usepackage{cite}

\definecolor{dkgreen}{rgb}{0,0.6,0}
\definecolor{gray}{rgb}{0.5,0.5,0.5}
\definecolor{daynineyellow}{rgb}{1.0,0.655,0.102}
\definecolor{url}{rgb}{0.1,0.1,0.4}

\lstset{frame=tb,
	language=Python,
	aboveskip=3mm,
	belowskip=3mm,
	showstringspaces=false,
	columns=flexible,
	basicstyle={\small\ttfamily},
	numbers=none,
	numberstyle=\tiny\color{gray},
	keywordstyle=\color{blue},
	commentstyle=\color{daynineyellow},
	stringstyle=\color{dkgreen},
	breaklines=true,
	breakatwhitespace=true,
	tabsize=3
}

\lstset{inputpath="C:/Users/Torstein/Documents/UiO/Mat1120/Python programmer"}
\hypersetup{colorlinks, urlcolor=url}

\author{Torstein Solheim Ølberg}
\title{Svar på Oblig nr. 1 i Mat1120}



%\lstinputlisting{Oblig1_1c.py}
%\includegraphics[width=12.6cm,height=8cm]{"C:/Users/Torstein/Documents/UiO/Mat1120/Python programmer"/Oblig1_1c.png}



\begin{document}
\maketitle
	\begin{center}
\Large \textbf{Oppgaver}
	\end{center}




		\paragraph{1.}
			\subparagraph{a)}
				\begin{flushleft}
Radreduserer matrisa og finner pivotelementene. De kolonene i matrise A som tilsvarer kollonene med pivoteelementer i matrise B er basisvektorene for kolonerommet til A. \\
Brukte programmet i oppg. b for å redusere matrisa.
				\end{flushleft}
				\begin{align}
A = \left[
\begin{tabular} { ccccc }
$1$ & $2$ & $-1$ & $3$ & $0$ \\
$2$ & $3$ & $-1$ & $5$ & $4$ \\
$1$ & $1$ & $0$ & $2$ & $1$ \\
$4$ & $2$ & $2$ & $6$ & $1$
\end{tabular}
\right] \sim B = \left[
\begin{tabular} { ccccc }
$1$ & $0$ & $1$ & $1$ & $0$ \\
$0$ & $1$ & $-1$ & $1$ & $0$ \\
$0$ & $0$ & $0$ & $0$ & $1$ \\
$0$ & $0$ & $0$ & $0$ & $0$
\end{tabular}
\right] \\
Basis_{kol} = 
\left\{
\left[
\begin{tabular} { c }
$1$ \\
$2$ \\
$1$ \\
$4$
\end{tabular}
\right]
,
\left[
\begin{tabular} { c }
$2$ \\
$3$ \\
$1$ \\
$2$
\end{tabular}
\right]
,
\left[
\begin{tabular} { c }
$0$ \\
$4$ \\
$1$ \\
$1$
\end{tabular}
\right]
\right\}
				\end{align}
				\begin{flushleft}
For radrommet har vi at de radene i B som ikke er bare $0$ere vil danne en basis for A 
				\end{flushleft}
				\begin{align}
Basis_{rad} = 
\left \{
\left[
\begin{tabular} { c }
$1$ \\
$0$ \\
$1$ \\
$1$ \\
$0$
\end{tabular}
\right]
,
\left[
\begin{tabular} { c }
$0$ \\
$1$ \\
$-1$ \\
$1$ \\
$0$
\end{tabular}
\right]
,
\left[
\begin{tabular} { c }
$0$ \\
$0$ \\
$0$ \\
$0$ \\
$1$
\end{tabular}
\right]
\right\}
				\end{align}
				\begin{flushleft}
Ranken til A er dimensjonen til kolonnerommet, altså antall pivotelementer i den radreduserte matrisa B.
				\end{flushleft}
				\begin{align}
rank(A) = 3
				\end{align}








			\subparagraph{b)}
				\begin{flushleft}
Bruker funksjonen nullspace i numpy for å finne nullrommet. Dimmensjonen til nullrommet er antall kollonner minus antall kolloner med et pivotelement.
				\end{flushleft}
\lstinputlisting{Oblig1_1a.py}
				\begin{align}
Basis_{Nul(A)} = 
\left \{
\left[
\begin{tabular} { c }
$-1$ \\
$1$ \\
$1$ \\
$0$ \\
$0$
\end{tabular}
\right]
,
\left[
\begin{tabular} { c }
$-1$ \\
$-1$ \\
$0$ \\
$1$ \\
$0$
\end{tabular}
\right]
\right\} \\
dim(Nul(A)) = 5 - 3 = 2
				\end{align}










		\paragraph{2.}
			\subparagraph{a)}
				\begin{flushleft}
Hvis $W = Span \{ \beta \}$ så sier definisjonen fra bok av en basis at hvis $\beta$ er lineært uavhengig så er $\beta$ en basis for $W$. Siden ingen av de funksjonene i $\beta$ kan skrives som en lineærkombinasjon av de andre, så er $\beta$ lineært  uavhengig og $\beta$ er dermed en basis for $W$
				\end{flushleft}
				\begin{align}
x = [0, \frac{\pi}{6}, \frac{\pi}{2}, \pi, 2 \pi]
\left[
\begin{tabular} { ccccc }
$1$ & $0$ & $1$ & $0$ & $1$ \\
$1$ & $\frac{1}{2}$ & $\frac{\sqrt{3}}{2}$ & $\frac{\sqrt{3}}{2}$ & \frac{1}{2} \\
$1$ & $1$ & $0$ & $0$ & $-1$ \\
$1$ & $0$ & $1$ & $0$ & $1$
\end{tabular}
\right]
\sim
\left[
\begin{tabular} { cccccc }
$1$ & $0$ & $1$ & $0$ & $1$ & $0$ \\
$1$ & $\frac{1}{2}$ & $\frac{\sqrt{3}}{2}$ & $\frac{\sqrt{3}}{2}$ & \frac{1}{2} & $0$ \\
$1$ & $1$ & $0$ & $0$ & $-1$ & $0$ \\
$1$ & $0$ & $-1$ & $0$ & $1$ & $0$ \\
$1$ & $0$ & $1$ & $0$ & $1$ & $0$
\end{tabular}
\right]
				\end{align}










			\subparagraph{b)}
				\begin{flushleft}
For at T skal være en lineæravbildning må den bestå disse testene:
				\end{flushleft}
				\begin{align}
T(f_1) + T(f_2) = T(f_1 + f_2) \\
T(f_1) + T(f_2) = \left(
\begin{tabular} { c }
$f_1(0)$ \\
$f_1(\frac{\pi}{2})$ \\
$2 f_1(0) - f_1(\frac{\pi}{2})$
\end{tabular}
\right)
+ 
\left(
\begin{tabular} { c }
$f_2(0)$ \\
$f_2(\frac{\pi}{2})$ \\
$2 f_2(0) - f_2(\frac{\pi}{2})$
\end{tabular}
\right) \nonumber \\
=
\left(
\begin{tabular} { c }
$f_1(0) + f_2(0)$ \\
$f_1(\frac{\pi}{2}) + f_2(\frac{\pi}{2})$ \\
$2 f_1(0) - f_1(\frac{\pi}{2}) + 2 f_2(0) - f_2(\frac{\pi}{2})$
\end{tabular}
\right)
= T(f_1 + f_2) \nonumber \\
cT(f) = T(cf) \\
cT(f) = c \left(
\begin{tabular} { c }
$f(0)$ \\
$f(\frac{\pi}{2})$ \\
$(2 f(0) - f(\frac{\pi}{2}))$
\end{tabular}
\right) \\
\left(
\begin{tabular} { c }
$f(0)$ \\
$f(\frac{\pi}{2})$ \\
$(2 f(0) - f(\frac{\pi}{2}))$
\end{tabular}
\right)
= 
\left(
\begin{tabular} { c }
$cf(0)$ \\
$cf(\frac{\pi}{2})$ \\
$c2 f(0) - f(\frac{\pi}{2})$
\end{tabular}
\right)
= T(cf)
				\end{align}







			\subparagraph{c)}
				\begin{flushleft}
Ikke besvart
				\end{flushleft}









			\subparagraph{d)}
				\begin{flushleft}
ikke besvart
				\end{flushleft}









			\subparagraph{e)}
				\begin{flushleft}
ikke besvart
				\end{flushleft}









			\subparagraph{f)}
				\begin{flushleft}
ikke besvart
				\end{flushleft}












		\paragraph{3.}
			\subparagraph{a)}
				\begin{align}
P = \left
[\begin{tabular} { ccc }
0.4 & 0.3 & 0.2 \\
0.5 & 0.5 & 0.2 \\
0.1 & 0.2 & 0.6
\end{tabular}
\right] \nonumber
				\end{align}












			\subparagraph{b)}
				\begin{flushleft}

				\end{flushleft}
				\begin{align}
P - I = \left[
\begin{tabular} {ccc}
-0.6 & 0.3 & 0.2 \\
0.5 & -0.5 & 0.2 \\
0.1 & 0.2 & -0.4
\end{tabular}
\right] \\
\text{radreduserer:} \nonumber \\
\left[
\begin{tabular} {ccc}
1 & 0 & -16/15 \\
0 & 1 & -22/15 \\
0 & 0 & -0
\end{tabular}
\right] \\
\text{likevektsvektoren blir da:} \nonumber \\
q = \left[
\begin{tabular} {c}
240/585 \\
330/585 \\
15/585
\end{tabular}\right] \\
lim_{n \rightarrow \infty} v_n = 53000 \cdot q = \left[
\begin{tabular} {c}
34051 \\
29897 \\
1359
\end{tabular}\right]
				\end{align}








			\subparagraph{c)}
\lstinputlisting{Oblig1_3c.py}
\end{document}