\documentclass[11pt, A4paper,norsk]{article}
\usepackage[utf8]{inputenc}
\usepackage[T1]{fontenc}
\usepackage{babel}
\usepackage{amsmath}
\usepackage{amsfonts}
\usepackage{amsthm}
\usepackage{amssymb}
\usepackage[colorlinks]{hyperref}
\usepackage{listings}
\usepackage{color}
\usepackage{hyperref}
\usepackage{graphicx}
\usepackage{cite}
\usepackage{textcomp}
\usepackage{float}

\definecolor{dkgreen}{rgb}{0,0.6,0}
\definecolor{gray}{rgb}{0.5,0.5,0.5}
\definecolor{daynineyellow}{rgb}{1.0,0.655,0.102}
\definecolor{url}{rgb}{0.1,0.1,0.4}

\lstset{frame=tb,
	language=Python,
	aboveskip=3mm,
	belowskip=3mm,
	showstringspaces=false,
	columns=flexible,
	basicstyle={\small\ttfamily},
	numbers=none,
	numberstyle=\tiny\color{gray},
	keywordstyle=\color{blue},
	commentstyle=\color{daynineyellow},
	stringstyle=\color{dkgreen},
	breaklines=true,
	breakatwhitespace=true,
	tabsize=3
}

\lstset{inputpath="C:/Users/Torstein/Documents/UiO/Fysmek1110/Python programmer"}
\graphicspath{{C:/Users/Torstein/Documents/UiO/Fysmek1110/"Python programmer"/}}
\hypersetup{colorlinks, urlcolor=url}

\author{Torstein Solheim Ølberg}
\title{Eksamen Fysmek1110 våren 2017}



%\lstinputlisting{Filnavn! type kodefil}
%\includegraphics[width=12.6cm,height=8cm]{Filnavn! type png}



\begin{document}
\maketitle
	\begin{center}
\Large \textbf{Oppgaver}
	\end{center}









		\paragraph{1.}
			\subparagraph{a)}
				\begin{gather*}
\text{Bevegelseesmengde bevart} \\
p_1 = p_2 \Rightarrow m_1 v_1 = m_2 v_2 \\
v_2 = \frac{m_1 \cdot v_1}{m_2} = \frac{20 \text{kg} \cdot 1 \text{m$/$s}}{25 \text{kg}} = \frac{4}{5} \text{m$/$s} = 0.8 \text{m$/$s}
				\end{gather*}









			\subparagraph{b)}
				\begin{flushleft}
Potensiell energi er ikke endret, men kinetisk er det.
$$E_1 = \frac{1}{2} m_1 v_1^2 = \frac{1}{2} \cdot 20 \text{kg} \cdot 1^2 \text{m$^2/$s$^2$} = 10 \text{J}$$
$$E_2 = \frac{1}{2} m_2 v_2^2 = \frac{1}{2} \cdot 25 \text{kg} \cdot 0.8^2 \text{m$^2/$s$^2$} = 8 \text{J}$$
Altså mister fisken $E_1 - E_2 = 10 \text{J} - 8 \text{J} = 2 \text{J}$
				\end{flushleft}

			








		\paragraph{2.}
			\subparagraph{a)}
				\begin{gather*}
x' = \frac{x}{\gamma} = x \sqrt{1 - \frac{u^2}{c^2}} \\
\left( \frac{x'}{x} \right)^2 = 1 - \frac{u^2}{c^2} \\
u = c \sqrt{1 - \left( \frac{x'}{x} \right)^2} \\
v_1 = c \sqrt{1 - \left( \frac{60}{100} \right)^2} = 0.8c \\
v_2 = c \sqrt{1 - \left( \frac{80}{100} \right)^2} = 0.6c
				\end{gather*}










			\subparagraph{b)}
				\begin{gather*}
v' = \frac{v - u}{1 - \frac{u}{c^2}v} \\
v'_2 = \frac{0.6c - 0.8c}{1 - \frac{0.6c \cdot 0.8c}{c^2}} = \frac{0.6 - 0.8}{1 - 0.6 \cdot 0.8}c = - 0.38 \\
v'_1 = \frac{0.8c - 0.6c}{1 - \frac{0.8c \cdot 0.6c}{c^2}} = \frac{0.8 - 0.6}{1 - 0.8 \cdot 0.6}c = 0.38
				\end{gather*}









		\paragraph{3.}
			\subparagraph{a)}
				\begin{flushleft}
Kreftene som inngår er $S$, $R$, $G$, $F$ og $N$
				\end{flushleft}









			\subparagraph{b)}
				\begin{gather*}
\Sigma F_x = 0 = F - R - S \sin (\theta) \\
\frac{1}{2} F - R = 0 \\
0 = F - \frac{1}{2} F - S \sin (\theta) = \frac{1}{2} F - S \sin (\theta) \\
S = \frac{F}{2 \sin(\theta)}
				\end{gather*}












			\subparagraph{c)}
				\begin{flushleft}
Hvis en stanga ikke skal bevege seg så må
$$F = 2 R = 2 \mu_s N = 2 \mu_s (mg + S \cos(\theta)) = 2 \mu_s \left( mg + \frac{F}{2 \sin(\theta)} \cos(\theta) \right)$$ \\
$$F \left( 1 - \frac{\mu_s}{\tan(\theta)} \right) = 2 \mu_s mg \Rightarrow F = \frac{2 \mu_s mg}{\left( 1 - \frac{\mu_s}{\tan(\theta)} \right)} $$
				\end{flushleft}











		\paragraph{4}
			\subparagraph{a)}
				\begin{flushleft}
Massesenteret er gitt ved $\vec{R} = \frac{1}{M}(m \cdot \frac{1}{2}L + m \cdot L) = \frac{\frac{3}{2}mL}{2m} = \frac{3}{4}L$
				\end{flushleft}










			\subparagraph{b)}
				\begin{flushleft}
Bruker Parallelakseteoremet:
$$I = I_{cm} + I_s + I_{bt} = \frac{1}{12}mL^2 + m \cdot \frac{1}{2}^2L^2 + mL^2 = \left( \frac{1}{12} + \frac{1}{4} + 1 \right)mL^2$$
$$\left( \frac{1 + 3 + 12}{12} \right)mL^2 = \frac{16}{12} mL^2 = \frac{4}{3} mL^2$$
				\end{flushleft}










			\subparagraph{c)}
				\begin{flushleft}
Bruker N2L for rotasjon
$$\tau = I \alpha = G_x \frac{3}{4} L = \frac{3}{4} 2mgL \sin(\theta)$$
$$\alpha = \frac{\frac{3}{2} mgL}{\frac{4}{3}mL^2} \sin(\theta) = \frac{9g}{8L} \sin(\theta)$$
				\end{flushleft}








			\subparagraph{d)}
				\begin{flushleft}
Bruker N2L for rotasjon
$$\tau = I \alpha = G_x \frac{1}{4} L = \frac{1}{2} mgL \sin(\theta)$$
Bruker Parallelakseteoremet:
$$I = I_{cm} + I_s + I_{bt} = \frac{1}{12}mL^2 + m \cdot \frac{1}{2}^2L^2 = \frac{1 + 3}{12}mL^2 = \frac{1}{3}mL^2$$
$$\alpha = \frac{\frac{1}{2} mgL}{\frac{1}{3}mL^2} \sin(\theta) = \frac{3g}{2L} \sin(\theta)$$
vinkelakselrasjonen er større nå en tidligere, altså vil staven falle fortere hvis blytuppen ligger på bordet
				\end{flushleft}









			\subparagraph{e)}
				\begin{flushleft}
Det er enklere å balansere staven med den tynne delen nederst fordi staven da vil bruke lengre tid på å falle enn hvis det var omvendt, noe som vil gjøre at du får bedre tid til å rette opp ubalansen.
				\end{flushleft}










		\paragraph{5.}
			\subparagraph{a)}
				\begin{flushleft}
Kreftene som virker er $F$, $G$ og $N$.
				\end{flushleft}









			\subparagraph{b)}
				\begin{gather*}
\text{Kraftmomentet om massesenteret, for alle kreftene, er gitt ved} \\
\vec{\tau} = \vec{F}R = F R \vec{z}
				\end{gather*}









			\subparagraph{c)}
				\begin{gather*}
\vec{\tau} = I \vec{\alpha} = FR \vec{z} = \frac{2}{5} m R^2 \left( - \frac{\vec{a}}{R} \right) \\
FR = - \frac{2}{5} m R \vec{a} \\
F = - \frac{2}{5} m \vec{a} \\
\Sigma F = F - G \sin(\theta) = ma \\
- G \sin(\theta) = ma + \frac{2}{5} ma = \frac{7}{5} ma \\
- \frac{5}{7} g \sin(\theta) = a
				\end{gather*}









			\subparagraph{d)}
				\begin{gather*}
\Sigma F = F - G \sin(\theta) = ma \\
F = - \frac{5}{7} mg \sin(\theta) + mg \sin(\theta) = \frac{2}{7} mg \sin(\theta)
				\end{gather*}









			\subparagraph{e)}
				\begin{gather*}
\text{For at kula skal rulle og ikke skli så må systemet tilfredstille rullebetingelsen} \\
\text{Da har vi fra tidligere at akselrasjonen er gitt ved} \\
a = - \frac{5F}{2m} \\
a = - \frac{5}{7} g \sin(\theta) \\
\frac{5}{2m} F = \frac{5}{7} g \sin(\theta) \\
- \mu_s g \cos(\theta) = \frac{2g}{7} \sin(\theta) \\
\mu_s = - \frac{2}{7} \tan(\theta)
				\end{gather*}









			\subparagraph{f)}
				\begin{flushleft}
Hvis starthastigheten er $v_0$ og akselrasjonen er som tidligere så er
$$v^2 = v_0^2 + 2 a d \Rightarrow d = \frac{v_0^2}{2 a} = \frac{7 v_0^2}{10 g \sin(\theta)}$$
Det vil si at distansen blir større med kvadratet av starthastigheten, og mindre jo større vinkelen $\theta$ er.
				\end{flushleft}








		\paragraph{6}
			\subparagraph{a)}
				\begin{flushleft}
Kreftene som er med er $R$ og $G$
				\end{flushleft}









			\subparagraph{b)}
				\begin{gather*}
r = \sqrt{y^2 + d^2} \\
F_{k} = - k (r - d) = - k (\sqrt{y^2 + d^2} - d) \\
F_{k,y} = F_k \sin(\theta) \\
F_{k,y} = - k (\sqrt{y^2 + d^2} - d) \frac{y}{\sqrt{y^2 + d^2}} \\
F_{k,y} = - k \left( y - \frac{yd}{\sqrt{y^2 + d^2}} \right) = - ky \left( 1 - \frac{d}{\sqrt{y^2 + d^2}} \right)
				\end{gather*}









			\subparagraph{c)}
				\begin{gather*}
r = \sqrt{y^2 + d^2} \\
F_{k} = - k (r - d) = - k (\sqrt{y^2 + d^2} - d) \\
F_{k,x} = F_k \cos(\theta) \\
F_{k,x} = - k (\sqrt{y^2 + d^2} - d) \frac{d}{\sqrt{y^2 + d^2}} \\
F_{k,x} = - k \left( d - \frac{d^2}{\sqrt{y^2 + d^2}} \right) = - kd \left( 1 - \frac{d}{\sqrt{y^2 + d^2}} \right)
				\end{gather*}









			\subparagraph{d)}
				\begin{flushleft}
Bruker uttryket for friksjonen, og at normalkraften må tilsvare fjerkraften i $x-$retning
$$R = - \mu_d N \frac{\vec{v_y}}{|v_y|} = \mu_d F_{k,x} \frac{\vec{v_y}}{|v_y|} = - \mu_d kd \left( 1 - \frac{d}{\sqrt{y^2 + d^2}} \right) \frac{\vec{v_y}}{|v_y|}$$
Brøken etter uttryket er for å sørge for at kraften alltid er motsatt av hastighetens retning.
				\end{flushleft}









			\subparagraph{e)}
				\begin{flushleft}
for i in xrange(len(t) - 1): \\
$\hspace{5mm} F\_ky = - k * y[i] * ( 1 - d / sqrt(y[i]^2 + d^2))$ \\
$\hspace{5mm} F\_kx = - k * d ( 1 - d / sqrt(y[i]^2 + d^2))$ \\
$\hspace{5mm} a = (F\_ky * linalg.norm(y) + mu\_d * F\_kx * linalg.norm(v[i]) - mg) / m$ \\
$\hspace{5mm} v[i + 1] = v[i] + a * dt$ \\
$\hspace{5mm} y[i + 1] = y[i] + v[i + 1] * dt$ \\
				\end{flushleft}









			\subparagraph{f)}
				\begin{flushleft}
Hvis du nå fester en ny fjær på den andre siden vil det ikke lenger være noen normalkraft som motstår fjærekraften i $x-$retning, og det blir derfor heller ikke noen friksjon. I virkeligheten vil det være litt friksjon siden det er umulig å sikke trekke sylinderen litt mer i en retning enn i den andre, men dette vil være en mye mindre friksjon enn tidligere. Siden vi ser bort i fra luftmotstand vil denne svigningen vare i teorien evig.
				\end{flushleft}
\end{document}