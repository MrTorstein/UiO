\documentclass[11pt, A4paper,norsk]{article}
\usepackage[utf8]{inputenc}
\usepackage[T1]{fontenc}
\usepackage{babel}
\usepackage{amsmath}
\usepackage{amsfonts}
\usepackage{amsthm}
\usepackage[colorlinks]{hyperref}
\usepackage{listings}
\usepackage{color}
\usepackage{hyperref}
\usepackage{graphicx}
\usepackage{cite}

\definecolor{dkgreen}{rgb}{0,0.6,0}
\definecolor{gray}{rgb}{0.5,0.5,0.5}
\definecolor{daynineyellow}{rgb}{1.0,0.655,0.102}
\definecolor{url}{rgb}{0.1,0.1,0.4}

\lstset{frame=tb,
	language=Python,
	aboveskip=3mm,
	belowskip=3mm,
	showstringspaces=false,
	columns=flexible,
	basicstyle={\small\ttfamily},
	numbers=none,
	numberstyle=\tiny\color{gray},
	keywordstyle=\color{blue},
	commentstyle=\color{daynineyellow},
	stringstyle=\color{dkgreen},
	breaklines=true,
	breakatwhitespace=true,
	tabsize=3
}

\lstset{inputpath="C:/Users/Torstein/Documents/UiO/Mat1120/Python programmer"}
\hypersetup{colorlinks, urlcolor=url}

\author{Torstein Solheim Ølberg}
\title{Svar på Oblig nr. 2 i Mat1120}



%\lstinputlisting{Filnavn! type kodefil}
%\includegraphics[width=12.6cm,height=8cm]{"C:/Users/Torstein/Documents/UiO/Mat1120/Python programmer"/Filnavn! type .png}



\begin{document}
\maketitle
	\begin{center}
\Large \textbf{Oppgaver}
	\end{center}









		\paragraph{1.}
			\subparagraph{a)}
				\begin{flushleft}
Radreduserer A for å finne hvilke kolloner som er lineært uavhengige. Lager Basisen av disse, og bruker Gram-Schmidts for å få en ortagonal basis.
				\end{flushleft}
				\begin{align}
W = Col(A) \\
A = 
\left[
\begin{tabular} { cccc }
1 & 1 & 0 & 2 \\
0 & 1 & 0 & 1 \\
1 & 0 & 0 & 1 \\
1 & 0 & 1 & 2
\end{tabular}
\right] 
\sim
\left[
\begin{tabular} { cccc }
1 & 0 & 0 & 1 \\
0 & 1 & 0 & 1 \\
0 & 0 & 1 & 1 \\
0 & 0 & 0 & 0
\end{tabular}
\right] \\
Basis_{Col(A)} =
\left\{
\left[ \begin{tabular} { c }
1 \\
0 \\
1 \\
1
\end{tabular}
\right] ,
\left[
\begin{tabular} { c }
1 \\
1 \\
0 \\
0
\end{tabular}
\right] ,
\left[
\begin{tabular} { c }
0 \\
0 \\
0 \\
1
\end{tabular}
\right]
\right\} \\
\vec{v_1} = \vec{x_1} \\
\vec{v_2} = \vec{x_2} - \frac{\vec{x_2} \cdot \vec{x_1}}{\vec{x_1} \cdot \vec{x_1}} \vec{x_1}  = 
\left[
\begin{tabular} { c }
2 \\
3 \\
-1 \\
-1
\end{tabular}
\right] \\
v_3 = x_3 - \frac{x_3 \cdot x_1}{x_1 \cdot x_1}x_1 - \frac{x_3 \cdot v_2}{v_2 \cdot v_2} =
\left[
\begin{tabular} { c }
-1 \\
1 \\
-2 \\
3
\end{tabular}
\right] \\
B = \left\{ 
\left[
\begin{tabular} { c }
1 \\
1 \\
0 \\
1
\end{tabular}
\right], \left[
\begin{tabular} { c }
2 \\
3 \\
-1 \\
-1
\end{tabular}
\right], \left[
\begin{tabular} { c }
-1 \\
1 \\
-2 \\
3
\end{tabular}
\right]
\right\}
				\end{align}








			\subparagraph{b)}
				\begin{flushleft}
finner $proj_W(y)$ og bruker denne til å lage en fjerde vektor som jeg legger til basisen $B$
				\end{flushleft}
				\begin{align}
y = 
\left[
\begin{tabular} { c }
1 \\
1 \\
1 \\
1
\end{tabular}
\right] \\
proj_W (y) = \frac{y \cdot v_1}{v_1 \cdot v_1}v_1 + \frac{y \cdot v_2}{v_2 \cdot v_2}v_2 + \frac{y \cdot v_3}{v_3 \cdot v_3}v_3 = \frac{1}{3} 
\left[
\begin{tabular} { c }
4 \\
2 \\
2 \\
3
\end{tabular}
\right] \\
v_4 = y - proj_W = \frac{1}{3}
\left[
\begin{tabular} { c }
-1 \\
1 \\
1 \\
0
\end{tabular}
\right] \\
C =
\left \{ 
\left[
\begin{tabular} { c }
1 \\
1 \\
0 \\
1
\end{tabular}
\right]
,
\left[
\begin{tabular} { c }
2 \\
3 \\
-1 \\
-1
\end{tabular}
\right]
,
\left[
\begin{tabular} { c }
-1 \\
1 \\
-2 \\
3
\end{tabular}
\right]
,
\left[
\begin{tabular} { c }
-1 \\
1 \\
1 \\
0
\end{tabular}
\right]
\right \}
				\end{align}








			\subparagraph{c)}
				\begin{flushleft}
Tester om $y$ er i Col(A), og siden den ikke er det så bruker jeg minste kvadraters metode for å finne x'
				\end{flushleft}
				\begin{align}
\begin{tabular} { c }
x + 1 = 1 \\
y = 1 \\
x = 1 \\
\end{tabular} \\
||y' - Ax'|| \leq ||y - Ax|| \\
y' = proj_{Col(A)}(y) = \frac{1}{3}
\left[
\begin{tabular} { c }
4 \\
2 \\
2 \\
3
\end{tabular}
\right] \\
Ax' = y'
\text{radreduserer $Ax'$ i python}
\begin{tabular} { c }
$x_1$ + $x_4$ = $\frac{2}{3}$ \\
$x_2$ + $x_4$ = $\frac{2}{3}$ \\
$x_3$ + $x_4$ = $\frac{1}{3}$ \\
0 = 0
\end{tabular} \\
x' = 
\left[
\begin{tabular} { c }
$\frac{2}{3}$ - $x_4$ \\
$\frac{2}{3}$ - $x_4$ \\
$\frac{1}{3}$ - $x_4$ \\
$x_4$
\end{tabular}
\right] = \frac{1}{3}
\left[
\begin{tabular} { c }
2 \\
2 \\
1 \\
0
\end{tabular}
\right]
+ x_4
\left[
\begin{tabular} { c }
-1 \\
-1 \\
-1 \\
1
\end{tabular}
\right]
				\end{align}
			









		\paragraph{2.}
			\subparagraph{a)}
				\begin{flushleft}
Finner det karakteristiske polynomet ved å regne ut $det(A/B - \lambda I)$
				\end{flushleft}
				\begin{align}
det \left|
\begin{tabular} { ccc }
$- \lambda$ &$ -3 $&$ -3$ \\
$-1$ & $2 - \lambda$ & $-1$ \\
$-2$ & $-2$ & $1 - \lambda$
\end{tabular}
\right| = - \lambda^3 + 3 \lambda^2 + 9 \lambda - 27 \\
det \left|
\begin{tabular} { ccc }
$- \lambda$ & 3 & $-6$ \\
$-1$ & $- \lambda$ & $0$ \\
$-2$ & $-6$ & $3 - \lambda$
\end{tabular}
\right| = - \lambda^3 + 3 \lambda^2 + 9 \lambda - 27
				\end{align}







			\subparagraph{b)}
				\begin{flushleft}
Finner en egenverdi for A ved hjelp av $\vec{v}$ og siden den finnes så vet jeg at v er en egenvektor. Derretter regner jeg ut basiser for egenromene på vanlig måte.
				\end{flushleft}
				\begin{align}
A \vec{v} = \lambda \vec{v} \\
\begin{tabular} { c }
-3 + 3 = 0 \\
2 + 1 = $\lambda$ \\
- 2 - 1 = - $\lambda$
\end{tabular} \Rightarrow \lambda = 3 \\
- \lambda^3 + 3 \lambda^2 + 9 \lambda - 27 \div (\lambda - 3) = - \lambda^2 + 9 = -(\lambda - 3)(\lambda + 3) \\
\lambda = \pm 3 \\
A \vec{x} = 3 \vec{x} \\
\vec{x} = y
\left[
\begin{tabular} { c }
-1 \\
1 \\
0
\end{tabular}
\right] + 
\left[
\begin{tabular} { c }
-1 \\
0 \\
1
\end{tabular}
\right] \\
A \vec{x} = -3 \vec{x} \\
\vec{x} = z
\left[
\begin{tabular} { c }
$\frac{3}{2}$ \\
$\frac{1}{2}$ \\
$1$
\end{tabular}
\right] \\
for A: Basis = 
\left \{
\left[
\begin{tabular} { c }
-1 \\
1 \\
0
\end{tabular}
\right]
,
\left[
\begin{tabular} { c }
-1 \\
0 \\
1
\end{tabular}
\right]
,
\left[
\begin{tabular} { c }
$\frac{3}{2}$ \\
$\frac{1}{2}$ \\
$1$
\end{tabular}
\right]
\right \} \\
B \vec{x} = 3 \vec{x}
\vec{x} = z
\left[
\begin{tabular} { c }
$\frac{1}{2}$ \\
$- \frac{3}{2}$ \\
$1$
\end{tabular}
\right] \\
B \vec{x} = -3 \vec{x}
\vec{x} = y
\left[
\begin{tabular} { c }
$3$ \\
$1$ \\
$0$
\end{tabular}
\right] \\
for B: Basis = 
\left \{
\left[
\begin{tabular} { c }
$\frac{1}{2}$ \\
$- \frac{3}{2}$ \\
$1$
\end{tabular}
\right]
,
\left[
\begin{tabular} { c }
3 \\
1 \\
0
\end{tabular}
\right]
\right \}
				\end{align}








			\subparagraph{c)}
				\begin{flushleft}
Siden P er den eneste av matrisene som har 3 egenvektorer så er denne den eneste som kan være diagonaliserbar.
				\end{flushleft}
				\begin{align}
P = 
\left[
\begin{tabular} { ccc }
$-1$ & $ -1$ & $\frac{3}{2}$ \\
$1$ & $0$ & $\frac{1}{2}$ \\
$0$ & $1$ & $1$
\end{tabular}
\right]
\text{og} D =
\left[
\begin{tabular} { ccc }
$3$ & $0$ & $0$ \\
$0$ & $3$ & $0$ \\
$0$ & $0$ & $-3$
\end{tabular}
\right]
				\end{align}
				\begin{flushleft}
Testet om AP og PD er like for å sjekke om P og D er riktig, og det var de.
				\end{flushleft}
				







		\paragraph{3.}
			\begin{align}
det \left| \begin{tabular} { ccc }
1 - $\lambda$ & 0 & $\sqrt{2}$ \\
0 & 2 - $\lambda$ & 0 \\
$\sqrt{2}$ & 0 & - $\lambda$
\end{tabular} \right| \\
= - \lambda^3 + 3 \lambda^2 - 4 \\
\text{Ser at $-1$ er en av egenverdiene, og bruker denne til å finne de andre} \nonumber \\
\lambda = - 1 \\
- \lambda^3 + 3 \lambda^2 - 4 \div (\lambda + 1) = - \lambda^2 + 4 \lambda - 4 \\
\text{Ser at dette er det samme som:} \nonumber \\
- (\lambda - 2)^2 \\
\text{Altså er $2$ også en egenverdi. sjekker om $\lambda = 2$ har flere egenvektorer} \nonumber \\
\left[ \begin{tabular} { ccc }
1 - $\lambda$ & 0 & $\sqrt{2}$ \\
0 & 2 - $\lambda$ & 0 \\
$\sqrt{2}$ & 0 & - $\lambda$
\end{tabular} \right]
\left[ \begin{tabular} { c }
x \\
y \\
z
\end{tabular} \right] = 2
\left[ \begin{tabular} { c }
x \\
y \\
z
\end{tabular} \right] \\
\begin{tabular} { c }
x + $\sqrt{2}$z = 2x \\
2y = 2y \\
$\sqrt{2}$x = 2z
\end{tabular}
			\end{align}
			\begin{align}
\begin{tabular} { c }
x = $\sqrt{2}$ z \\
y = y \\
$\sqrt{2}$ x = 2z
\end{tabular} \\
\text{Ser at y og z er frie variabler, egenverdiene blir:} \nonumber \\
\left[ \begin{tabular} { c }
$\sqrt{2}$ \\
0 \\
1
\end{tabular} \right], 
\left[ \begin{tabular} { c }
0 \\
1 \\
0
\end{tabular} \right]
			\end{align}
			\begin{flushleft}
Siden A har en basis som ikke er for et nullrom og denne basisen består av 3 egenvektorer, så er A ortagonalt diagonaliserbar. Finner den siste egenvektoren og så en ortagonal basis. Deretter finner jeg P og D.
			\end{flushleft}
			\begin{align}
\begin{tabular} { c }
x + $\sqrt{2}$ = - x \\
2y = - y \\
$\sqrt{2}$x = -z
\end{tabular} \\
\text{Da blir egenvektoren for $\lambda = -1$:} \nonumber \\
\left[ \begin{tabular} { c }
1 \\
0 \\
- $\sqrt{2}$
\end{tabular} \right] \\
B = \left\{
\left[ \begin{tabular} { c }
1 \\
0 \\
- $\sqrt{2}$
\end{tabular}
\right] ,
\left[
\begin{tabular} { c }
0 \\
1 \\
0
\end{tabular}
\right] ,
\left[
\begin{tabular} { c }
$\sqrt{2}$ \\
0 \\
1
\end{tabular}
\right]
\right\} \\
\text{Sjekker om denne basisen allerede er ortagonal:} \nonumber \\
v_1 \cdot v_2 =
\left[ \begin{tabular} { c }
1 \\
0 \\
- $\sqrt{2}$
\end{tabular} \right] \cdot
\left[ \begin{tabular} { c }
0 \\
1 \\
0
\end{tabular} \right] = 0 \\
v_2 \cdot v_3 =
\left[ \begin{tabular} { c }
0 \\
1 \\
0
\end{tabular} \right] \cdot
\left[ \begin{tabular} { c }
$\sqrt{2}$ \\
0 \\
1
\end{tabular} \right] = 0 \\
v_1 \cdot v_3 =
\left[ \begin{tabular} { c }
1 \\
0 \\
- $\sqrt{2}$
\end{tabular} \right] \cdot
\left[ \begin{tabular} { c }
$\sqrt{2}$ \\
0 \\
1
\end{tabular} \right] = 0
			\end{align}
			\begin{flushleft}
Da vet vi at basisen $B$ allerede er ortagonal og vi trenger ikke finne en ny basis. Finner så P og D ved å vite at P er egenvektorene og D er en diagonalmatrise med egenverdiene på diagonalen.
			\end{flushleft}
			\begin{align}
P = \left[ 
\begin{tabular} { ccc }
1 & 0 & $\sqrt{2}$ \\
0 & 1 & 0 \\
- $\sqrt{2}$ & 0 & 1
\end{tabular} 
\right] \\
D = \left[ 
\begin{tabular} { ccc }
-1 & 0 & 0 \\
0 & 2 & 0 \\
0 & 0 & 2
\end{tabular} 
\right]
			\end{align}






		\paragraph{4.}
			\subparagraph{a)}
				\begin{flushleft}
Lager en funksjon som tilpasser datasettet, gitt en funksjon for hvordan kurven skal se ut, og et datasett. Fikk funksjonene under
				\end{flushleft}
				\begin{align*}
funk: y = 3.05 + 17.84 x - 69.55 x^2
				\end{align*}












			\subparagraph{b)}
				\begin{flushleft}
Lager en ny form på kurva, og kaller på den samme funksjonen som i (a). Får da funksjonen under:
				\end{flushleft}
				\begin{align*}
funk: y = 2.96 \sin(2 \pi x) + 3.01 \cos(2\pi x)
				\end{align*}











			\subparagraph{c)}
				\begin{flushleft}
Regner ut de to normene som gitt i oppgaven, og bruker numpys innebygde funksjon for å regne ut normen til en vektor. Ut i fra svarene under velger jeg sinus/cosinus kurva som den beste tilnærmingen.
				\end{flushleft}
				\begin{align}
\epsilon_i = y_i - f(x_i) \nonumber \\
\text{for tredjegrads kurva får jeg normen:} \nonumber \\
|| \epsilon || = 0.0452961133406 \nonumber \\
\text{for sinus/cosinus kurva får jeg normen:} \nonumber \\
|| \epsilon || = 0.137044526055 \nonumber
				\end{align}
\end{document}