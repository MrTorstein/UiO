\documentclass[11pt, A4paper,norsk]{article}
\usepackage[utf8]{inputenc}
\usepackage[T1]{fontenc}
\usepackage{babel}
\usepackage{amsmath}
\usepackage{amsfonts}
\usepackage{amsthm}
\usepackage{amssymb}
\usepackage[colorlinks]{hyperref}
\usepackage{listings}
\usepackage{color}
\usepackage{hyperref}
\usepackage{graphicx}
\usepackage{cite}
\usepackage{textcomp}

\definecolor{dkgreen}{rgb}{0,0.6,0}
\definecolor{gray}{rgb}{0.5,0.5,0.5}
\definecolor{daynineyellow}{rgb}{1.0,0.655,0.102}
\definecolor{url}{rgb}{0.1,0.1,0.4}

\lstset{frame=tb,
	language=Python,
	aboveskip=3mm,
	belowskip=3mm,
	showstringspaces=false,
	columns=flexible,
	basicstyle={\small\ttfamily},
	numbers=none,
	numberstyle=\tiny\color{gray},
	keywordstyle=\color{blue},
	commentstyle=\color{daynineyellow},
	stringstyle=\color{dkgreen},
	breaklines=true,
	breakatwhitespace=true,
	tabsize=3
}

\lstset{inputpath="C:/Users/Torstein/Documents/UiO/Fys3140/Python programmer"}
\graphicspath{{C:/Users/Torstein/Documents/UiO/Fys3140/"Python programmer"/}}
\hypersetup{colorlinks, urlcolor=url}

\author{Torstein Solheim Ølberg}
\title{Svar på Oblig nr. 3 i Fys3140}



%\lstinputlisting{Filnavn! type kodefil}
%\includegraphics[width=12.6cm,height=8cm]{Filnavn! type png}



\begin{document}
\maketitle
	\begin{center}
\Large \textbf{Oppgaver}
	\end{center}









		\paragraph{1.}
			\subparagraph{a)}
				\begin{flushleft}
Siden integralet $\oint_{\Gamma} \frac{sin(z)dz}{2z - \pi}$ har en singulæritet inni sirkelen $|z| = 3$, altså $z = \frac{\pi}{2}$ vil dette integralet ikke være mulig å bruke Cauchy's teorem på. Integralet kan derimot også skrives på formen 
$$\oint_{\Gamma} \frac{\frac{sin(z)}{2}dz}{z - \frac{\pi}{2}} = \frac{1}{2} \oint_{\Gamma} \frac{\sin(z)dz}{z - \frac{\pi}{2}}$$
Da sier Cauchy's integral formel at
$$\pi i \sin\left(\frac{\pi}{2}\right) =  \frac{1}{2} \oint_{\Gamma} \frac{\sin(z)dz}{z - \frac{\pi}{2}}$$
Som vi kan se betyr at integralet blir lik $\pi i$
				\end{flushleft}









			\subparagraph{b)}
				\begin{flushleft}
Integralet $\oint_{\Gamma} \frac{sin(z)dz}{2z - \pi}$ har derimot ikke noen singulæriteter inni sirkelen $|z| = 1$ og Cauchy's teorem sier da at integralet blir lik null
				\end{flushleft}









			\subparagraph{c)}
				\begin{flushleft}
Integralet $\oint_{\Gamma} \frac{sin(z)dz}{6z - \pi}$ har en singulæritet inni sirkelen $|z| = 1$, altså $z = \frac{\pi}{6}$ vil dette integralet ikke være mulig å bruke Cauchy's teorem på heller. Integralet kan derimot også skrives på formen
$$\oint_{\Gamma} \frac{\frac{sin(z)}{6}dz}{z - \frac{\pi}{6}} = \frac{1}{2} \oint_{\Gamma} \frac{\frac{\sin(z)}{3}dz}{z - \frac{\pi}{6}}$$
Da sier Cauchy's integral formel at
$$\pi i \frac{\sin\left(\frac{\pi}{6}\right)}{3} =  \frac{1}{2} \oint_{\Gamma} \frac{\frac{\sin(z)}{3}dz}{z - \frac{\pi}{6}} = \pi i \frac{1}{2} \cdot \frac{1}{3} = i \frac{\pi}{6}$$
				\end{flushleft}
			









			\subparagraph{d)}
				\begin{flushleft}
Integralet $\oint_{\Gamma} \frac{e^{2z}dz}{z - \ln(2)}$ har singulæriteten $z = \ln(2)$ som ligger inni firkanten i oppgave teksten, altså en med senter i $0$ og sider med lengde $2$. Dermed kan vi ikke bruke Cauchy's teorem, men vi kan bruke Cauchy's integral formel direkte, som sier
$$2 \pi i e^{2 \ln(2)} = \oint_{\Gamma} \frac{e^{2z}dz}{z - ln(2)} = 2 \pi i (2)^2 = 8 \pi i$$
				\end{flushleft}










		\paragraph{2.}
			\subparagraph{a)}
				\begin{flushleft}
Deriverer Cauchys integral formel
				\end{flushleft}
				\begin{gather*}
f(z) = \frac{1}{2 \pi i} \oint_{\Gamma} \frac{f(w) dw}{( w - z)} \\
f'(z) = \frac{1}{2 \pi i} \oint_{\Gamma} \frac{f(w) dw}{(w - z)^{2}} \cdot - 1 \cdot - 1 = \frac{1}{2 \pi i} \oint_{\Gamma} \frac{f(w) dw}{(w - z)^{2}} \\
f''(z) = \frac{1}{2 \pi i} \oint_{\Gamma} \frac{f(w) dw}{(w - z)^{2}} \cdot - 2 \cdot - 1 = \frac{2}{2 \pi i} \oint_{\Gamma} \frac{f(w) dw}{(w - z)^{3}} \\
f'''(z) = \frac{2}{2 \pi i} \oint_{\Gamma} \frac{f(w) dw}{(w - z)^{3}} \cdot - 3 \cdot - 1 = \frac{6}{2 \pi i} \oint_{\Gamma} \frac{f(w) dw}{(w - z)^{4}}
				\end{gather*}
				\begin{flushleft}
Ser at alt som endrer seg for hver gang du deriverer er at du ganger med $n$, der $n$ er det tallet du deriverer med, og legger til en i potensen til telleren. Altså blir formelen 
$$f^{(n)}(z) = \frac{n!}{2 \pi i} \oint_{\Gamma} \frac{f(w) dw}{(w - z)^{n + 1}}$$
				\end{flushleft}








			\subparagraph{b)}
				\begin{flushleft}
Integralet $\oint_{\Gamma} \frac{\sin(2z)dz}{(6z - \pi)^3}$ er på en form som gjør at vi kan bruke den generaliserte Cauchy integral formelen med $n = 2$, altså
$$f^{(2)}(a) = \frac{2}{2 \pi i} \oint_{\Gamma} \frac{f(z) dz}{(z - a)^{3}} = \frac{1}{\pi i} \oint_{\Gamma} \frac{f(z) dz}{(z - a)^{3}}$$
Siden integralet vårt kan skrives på formen
$$\oint_{\Gamma} \frac{\sin(2z)dz}{\left( 6 \left(z - \frac{\pi}{6} \right) \right)^3} = \oint_{\Gamma} \frac{\frac{\sin(2z)}{\sqrt[3]{6}}dz}{\left(z - \frac{\pi}{6} \right)^3}$$
Altså får vi, ved hjelp av integralformelen
$$\oint_{\Gamma} \frac{\frac{\sin(2z)}{\sqrt[3]{6}}dz}{\left(z - \frac{\pi}{6} \right)^3} = \pi i \left( \frac{\sin(2z)}{\sqrt[3]{6}} \right)''\left( \frac{\pi}{6} \right) = \pi i \left( \frac{- 4 \sin\left( \frac{\pi}{3} \right)}{\sqrt[3]{6}} \right) = - i 2^{\frac{2}{3}} \sqrt[6]{3} \pi $$
				\end{flushleft}








		\paragraph{3.}
			\subparagraph{a)}
				\begin{gather*}
\oint_{\Gamma} \frac{\cosh(z)dz}{2 \ln (2) - z} = \oint_{\Gamma} \frac{- \cosh(z)dz}{z - 2 \ln (2)} \\
\text{Dette blir etter Cauchys integral formel} \\
\oint_{\Gamma} \frac{- \cosh(z)dz}{z - 2 \ln (2)} = - 2 \pi i \cosh(2 \ln(2)) = - 2 \pi i \frac{e^{2 \ln(2)} + e^{-2 \ln(2)}}{2} = - \frac{9 \pi i}{4}
				\end{gather*}









			\subparagraph{b)}
				\begin{gather*}
\oint_{\Gamma} \frac{e^{3z}dz}{\left( z - \ln(2) \right)^4} \\
\text{Dette blir etter Cauchys integral formel} \\
\oint_{\Gamma} \frac{e^{3z}dz}{\left( z - \ln(2) \right)^4} = \frac{2 \pi i \left( e^{3z} \right)'''(\ln(2))}{6} = 3 \pi i e^{3 \ln(2)} = 24 \pi i
				\end{gather*}











		\paragraph{4.}
			\subparagraph{a)}
				\begin{gather*}
\text{Utrykket} f(z) = \frac{z - 1}{z^2 (z - 2)} \text{kan også skrives som} \\
f(z) = \frac{1}{z^2}\left( \frac{z}{z - 2} - \frac{1}{z - 2} \right) \\
\text{Skriver på formen $\frac{1}{1 - \frac{z}{k}}$} \\
\frac{z}{z - 2} = - \frac{z}{2} \frac{1}{1 - \frac{z}{2}} = - \frac{z}{2} \sum_{n = 0}^{\infty} \left( \frac{z}{2} \right)^n = - \sum_{n = 0}^{\infty} \left( \frac{z}{2} \right)^{n + 1} \\
\text{Som vi kan se at konvergerer med $|z| < 2$} \\
\frac{1}{z - 2} = - \frac{1}{2} \frac{1}{1 - \frac{1}{2}} = - \sum_{n = 0}^{\infty} \frac{z^n}{2^{n + 1}} \\
\text{Dette konvergerer også når $|z| < 2$. Putter vi alt sammen får vi.} \\
f(z) = \frac{z - 1}{z^2(z - 2)} = \frac{1}{z^2} \left( \frac{z}{z - 2} - \frac{1}{z - 2} \right) = \frac{1}{z^2} \left( - \sum_{n = 0}^{\infty} \left( \frac{z}{2} \right)^{n + 1} - \sum_{n = 0}^{\infty} \frac{z^n}{2^{n + 1}} \right) \\
f(z) = - \sum_{n = 0}^{\infty} \left( \frac{\frac{1}{z} + \frac{1}{z^2}}{2^{n + 1}} \right)z^n \\
\text{Dette blir da Laurent serien for $|z| < 2$}
				\end{gather*}












			\subparagraph{b)}
				\begin{gather*}
f(z) = \frac{1}{z^2}\left( \frac{z}{z - 2} - \frac{1}{z - 2} \right) \\
\text{Skriver på formen $\frac{1}{1 - \frac{k}{z}}$} \\
\frac{z}{z - 2} = - \frac{1}{1 - \frac{2}{z}} = - \sum_{n = 0}^{\infty} \left( \frac{2}{z} \right)^n \\
\text{Som vi kan se at konvergerer med $|z| > 2$} \\
\frac{1}{z - 2} = - \frac{1}{z} \frac{1}{1 - \frac{2}{z}} = - \sum_{n = 0}^{\infty} \frac{2^{n}}{z^{n + 1}} \\
\text{Dette konvergerer også når $|z| > 2$. Putter vi alt sammen får vi.} \\
f(z) = \frac{z - 1}{z^2(z - 2)} = \frac{1}{z^2} \left( \frac{z}{z - 2} - \frac{1}{z - 2} \right) = \frac{1}{z^2} \left( - \sum_{n = 0}^{\infty} \left( \frac{2}{z} \right)^n - \sum_{n = 0}^{\infty} \frac{2^n}{z^{n + 1}} \right) \\
f(z) = - \sum_{n = 0}^{\infty} \left( \frac{2^n(z + 1)}{z^3} \right)\frac{1}{z^n} \\
\text{Dette blir da Laurent serien for $|z| > 2$}
				\end{gather*}









			\subparagraph{c)}
				\begin{flushleft}
Fra uttrykket for $f(z)$ gyldig innenfor sirkelen $|z| < 2$ har vi at $b_1$ er 
$$- \frac{1}{2} - \frac{1}{4} = - \frac{3}{4}$$
Hvor jeg henter den første brøken fra brøk en i telleren ved ledd $n = 0$ og den andre brøken fra brøk to i telleren, når $n = 1$
				\end{flushleft}
\end{document}