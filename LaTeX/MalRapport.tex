\documentclass[11pt, A4paper, english]{article}
\usepackage[utf8]{inputenc}
\usepackage[T1]{fontenc}
\usepackage{babel}
\usepackage{amsmath}
\usepackage{amsfonts}
\usepackage{amsthm}
\usepackage[colorlinks]{hyperref}
\usepackage{listings}
\usepackage{color}
\usepackage{hyperref}
\usepackage{graphicx}
\usepackage{cite}
\usepackage{float}
\usepackage{multicol}

\definecolor{dkgreen}{rgb}{0,0.6,0}
\definecolor{gray}{rgb}{0.5,0.5,0.5}
\definecolor{daynineyellow}{rgb}{1.0,0.655,0.102}
\definecolor{url}{rgb}{0.1,0.1,0.4}

\lstset{
frame=tb,
language=Python,
aboveskip=3mm,
belowskip=3mm,
showstringspaces=false,
columns=flexible,
basicstyle={\small\ttfamily},
numbers=none,
numberstyle=\tiny\color{gray},
keywordstyle=\color{blue},
commentstyle=\color{daynineyellow},
stringstyle=\color{dkgreen},
breaklines=true,
breakatwhitespace=true,
tabsize=3
}

%The exemples under is for paths in windows
\lstset{inputpath="Where you get your code files from (f.ex: C:/Users/Torstein/Documents/UiO/Ast2210/Python programmer)"}
\graphicspath{{Where you get your grafics files from (f.eks: C:/Users/Torstein/Documents/UiO/Faget!/"Python programmer"/)(Folders with whitespace in their name have to be written inside apostrophies)}}
\hypersetup{colorlinks, urlcolor=url}

%This is how to put in codefiles
%\lstinputlisting{Filnavn! type kodefil}

%This is how to put in pictures
%\includegraphics[width=12.6cm,height=8cm]{Filnavn! type png}


\author{
Your Name! \\
Name of institute!, University! \\
Mailing address!, Country
%F.eks:
%Institute for Theoretical Astrophysics, University of Oslo \\
%P.O. Box 1029 Blindern 0315 Oslo, Norway
}
\title{Title of the report!}

\begin{document}

\maketitle

	\section{Abstract}
%Here the article is briefly summarized, mentioning some background, the methods and data used as well as notable results - keep it short and to the point.


	\begin{multicols}{2}
		\section{Introduction}
%Why are we doing this exercise, what are our assumptions, what do we want to accomplish?

		\section{Data Description}

		\section{Methods}
%How did you obtain the data, what methods did you use? Describe your work process. If you performed an experiment using specific equipment, describe your setup.

	\end{multicols}

		\section{Results}
%What were your clear-cut results? - Present them in a clear and concise manner, waiting with the discussion of them for later. Here you present calculations, figures and tables of data, output of code, etc.
			\begin{tabular}{|l|r|}
\hline
Measured & Data \\
\hline
%First ting mesured! & First mesurement! \\
\hline
			\end{tabular}

	\begin{multicols}{2}
		\section{Conclusion/Discussion}
%Were the results what you expected? Are the results significant? - meaning; are the results clear, or are they open to interpretation? How certain can you be of them? What do these results mean in a wider context?


\addcontentsline{toc}{chapter}{Bibliography}
		\begin{thebibliography}{9}
%			\bibitem{Name of referance!}
%Name of Author, Name of Author, ...!; \\
%Title of article! \\
%\url{URL!} \\
%Year of release! \\
%Downloaded Date you downloaded webpage!
%			\bibitem{Name of referance!}
%Name of Author, Name of Author, ...!; \\
%Title of article!, Pages used! \\
%Year of release!, Town published!: Publisher! \\
		\end{thebibliography}
	\end{multicols}
\end{document}