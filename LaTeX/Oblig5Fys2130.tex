\documentclass[11pt, A4paper,norsk]{article}
\usepackage[utf8]{inputenc}
\usepackage[T1]{fontenc}
\usepackage{babel}
\usepackage{amsmath}
\usepackage{amsfonts}
\usepackage{amsthm}
\usepackage{amssymb}
\usepackage[colorlinks]{hyperref}
\usepackage{listings}
\usepackage{color}
\usepackage{hyperref}
\usepackage{graphicx}
\usepackage{cite}
\usepackage{textcomp}
\usepackage{float}

\definecolor{dkgreen}{rgb}{0,0.6,0}
\definecolor{gray}{rgb}{0.5,0.5,0.5}
\definecolor{daynineyellow}{rgb}{1.0,0.655,0.102}
\definecolor{url}{rgb}{0.1,0.1,0.4}

\lstset{frame=tb,
	language=Python,
	aboveskip=3mm,
	belowskip=3mm,
	showstringspaces=false,
	columns=flexible,
	basicstyle={\small\ttfamily},
	numbers=none,
	numberstyle=\tiny\color{gray},
	keywordstyle=\color{blue},
	commentstyle=\color{daynineyellow},
	stringstyle=\color{dkgreen},
	breaklines=true,
	breakatwhitespace=true,
	tabsize=3
}

\lstset{inputpath="C:/Users/Torstein/Documents/UiO/Fys2130/Python programmer"}
\graphicspath{{C:/Users/Torstein/Documents/UiO/Fys2130/"Python programmer"/}}
\hypersetup{colorlinks, urlcolor=url}

\author{Torstein Solheim Ølberg}
\title{Svar på Oblig nr. 5 i Fys2130}



%\lstinputlisting{Filnavn! type kodefil}
%\includegraphics[width=12.6cm,height=8cm]{Filnavn! type png}



\begin{document}
\maketitle
	\begin{center}
\Large \textbf{Oppgaver}
	\end{center}









		\paragraph{8.}
			\subparagraph{1)}
				\begin{flushleft}
Med et dispersivt medie menes et medie som har forskjellige hastigheter, for en bølge som brer seg i mediet, avhengig av bølgelengden til bølgen som brer seg i mediet. \\
Dispersjon vil ikke påvirke en harmonsik bølge i det hele tatt, fordi denne bølgen har en spesifikk bølgelngde, og derfor ikke vil bli brutt i flere bølger med forskjellige hastigheter. \\
ikke-harmoniske bølger kan derimot brytes opp i flere forskjellige bølger med forskjellige bølgelengder, og vil derfor splittes opp ved hjelp av dispersjon.
				\end{flushleft}









			\subparagraph{7)}
				\begin{flushleft}
Vi bruker en bølgepakke i beregningene fordi en bølge i virkeligheten har en start og en slutt og en spesifik bredde.
				\end{flushleft}









			\subparagraph{9)}
				\begin{flushleft}
Bølger på vannoverflaten er transversale bølger fordi de brer seg i horisontal rettning, men har et utslag som er i loddrett rettning. altså er utbredelsesrettning og utslagsrettning vinkelrett på hverandre, noe som er definisjonen av transversal bølge.
				\end{flushleft}

			








		\paragraph{9.}
			\subparagraph{1)}
				\begin{flushleft}
Et sted i rommet der divergensen til et elektrisk felt er forskjellig fra null er karakterisert ved at det strømmer elektriske felt linjer ut av eller inn til det punktet. Det vil si at en testladning enten vi bli frastødt eller tiltrukket det punktet. \\
Et sted i rommet der det er et elektrisk felt og dette har en rotasjon ulik null, kan karakteriseres med at det vil være en endring i det magnetiske feltet der.
				\end{flushleft}










			\subparagraph{9)}
				\begin{flushleft}
kompasset vil stille seg inn etter magnetfeltet til laseren istedenfor til det magnetiske feltet til jorda.
				\end{flushleft}









			\subparagraph{13)}
				\begin{flushleft}
Nei, det finnes ikke, fordi det ikke finnes noen løsninger av bølgelikningene som ikke også får en sammenhengen som er større enn dette når den skal passe inn med Maxwells lover.
				\end{flushleft}









		\paragraph{10.}
			\subparagraph{2)}
				\begin{flushleft}
Når vi ser refleksjon i et vindu vil vi kunne se to bilder fordi lyset først kan bli reflektert når det går fra luft til glass, men lyset kan også bli reflektert når det skal gå fra glass til luft på innsiden av ruta. \\
Jeg har ikke sett at det har dukket opp flere bilder enn to av og til.
				\end{flushleft}









			\subparagraph{3)}
				\begin{flushleft}
Du kan oppnå total refleksjon så lenge innfallsvinkelen er større enn $\arcsin\left( \frac{n_2}{n_1} \right)$ og $n_1 > n_2$. Da vil brytingsvinkelen være større enn infallsvinkelen, og hvis du fortsetter å øke innfallsvinkelen vil til slutt ikke brytningsvinkelen kunne være lavere enn $90$ grader. Siden du fortsatt kan øke innfalsvinkelen vil det da måtte bli null brytning.
				\end{flushleft}












			\subparagraph{4)}
				\begin{flushleft}
Lyset fra skjermen på smart telefonen din er polarisert, så hvis du kan rotere den på en måte som gjør at du ikke lenger ser noe serlig på skjermen vet du at solbrillene du har på deg er "av polaroidtypen".
				\end{flushleft}












			\subparagraph{7)}
				\begin{flushleft}
Vi kan ikke egentlig bruke fermats prinsipp fordi det utelukkende gjelder for lys. Men hvis vi antar at det samme prinsippet gjelder for lydbølger også er det mulig å forklare fenomenet. På dagen vil da nemlig lydbølgene velge å bevege seg langs bakke, der mye av lyden vil bli absorbert i bakken, mens om natten vil lyden velge å gå lenger oppe i lufta og det vil derfor ikke bli absorbert like fort i bakken.
				\end{flushleft}








		\paragraph{8.}
			\subparagraph{14.a)}
				\begin{flushleft}
Bølgen beveger seg nå fra midten, og mot venstre. Når den treffer en av sidene faller bølgen ned og får et utslag med motsatt fortegn, men samme størrelse. Dette er til forskjell fra før jeg endret fortegn på den deriverte av utslaget, da bølgen gikk mot høyre og ikke mot venstre.
				\end{flushleft}










			\subparagraph{14.b)}
				\begin{flushleft}
Endret nå fortegnet tilbake til slik det var i starten, men istedenfor delte jeg den tidsderiverte  av utslaget på to. Da får jeg at en stor del av bølgen går mot høyre som tidligere, men en litt mindre del går mot venstre. Ellers er oppførselen av bølgene lik som hvis de hadde vært samlet.
				\end{flushleft}








			\subparagraph{14.c)}
				\begin{flushleft}
Denne gangen dobler jeg den tidsderiverte. Da får jeg at bølgen igjen splittes i to deler. Også likt som tidligere er at den som beveger seg mot høyre er størst, og den som er minst begynner mot venstre. til forskjell fra tidligere er det derimot slik at de to bølgene har forskjellig fortegn. den lille bølgen har et negativt utslag, mens den store bølgen har et positivt utslag. Den store bølgen er også større enn det den har vært tidligere.
				\end{flushleft}









			\subparagraph{14.d)}
				\begin{flushleft}
Tenkte at den skulle settes opp som en sinus eller kosinus versjon av samme funksjonen $u$ som vi har i de tidligere oppgavene, men det fører bare til en slags kombinasjon av en bevegende sinusblkge og en stående sinusbølge.
				\end{flushleft}









			\subparagraph{14.e)}
				\begin{flushleft}
Nei, det er ett forhold mellom hvilke posisjoner og hvilke hastigheter som fungerer, gitt av hvordan bølgen ser ut og hvilke parameter som er gitt til den. Dette kommer av at bølgene naturlig ønsker å oppnå en spesifikk hastighet.
				\end{flushleft}







		\paragraph{9.}
			\subparagraph{19)}
				\begin{flushleft}
Antar at dette er snakk om en bølge som befinner seg langt unna kilden, og også langt unna frie strømmer og ladninger. \\
Bølgelengden til en planbølge er gitt ved $\lambda = \frac{2 \pi}{k}$. Antar $k$ i vårt tilfelle er lengden av $\vec{k}$ altså lik $1$.
$$\lambda = \frac{2 \pi}{k} = \frac{2 \cdot \pi}{1} = 2 \pi$$
Når det kommer til rettningen til bølgen så vet vi at $\vec{E}$ er i $z$ retning, og ut fra dette vet vi da også at positiv retning for $B$ er i $x$ retning. Da vet vi at selve bølgen er på vei i $y$ retning. For å finne $\vec{B}$ kan vi resultatet fra boka på side $236$ om at $\vec{B}$ da blir
$$\vec{B} = B_0 \cos(k y - \omega t) \vec{i} = c E_0 \cos(k y - \omega t) \vec{i}$$
				\end{flushleft}












			\subparagraph{25)}
				\begin{flushleft}
Hvis Vi bruker Faraday-Henrys lov på integral form kan vi se på en faraday løkke som går langs det elektriske feltet, og så vinkelrett på det og ut av det. For så å gå parallelt med det elektriske feltet og normalt på det tilbake inn igjen. Da skal det magnetiske feltet gjennom denne løkka endre seg med $- 1.2 \text{mA} \mu_r \mu_0 \cdot 9 \cdot 10^{8} 1/\text{s} \approx 9 \mu_0 \mu_r \cdot 10^{5} \text{V}/\text{m}$. Altså må enten $\mu_0 \mu_r$ er noe i størrelsesorden $10^{-5}$ eller så er disse helt ute av proposjon.
				\end{flushleft}










			\subparagraph{29)}
				\begin{flushleft}
Intensitet er gitt ved effekt per areal. Hvis vi antar en isotrop intensitet, vil det si at vi antar at intensiteten er lik i et kuleskall rundt telefonen. Arealet av dette kuleskallet er lik $4 \pi r^2$ og hvis vi skal se på intensiteten $5$cm unna mobilen, blir arealet $4 \pi \cdot 5^2 \text{cm}^2 = 10^{-2} \pi$. Altså blir intensiteten et sted mellom $\frac{7}{\pi} 10^{-3} \text{W}/\text{m}^2$ og $\frac{1}{\pi} 10^{-2} \text{W}/\text{m}^2$.
				\end{flushleft}









			\subparagraph{31)}
				\begin{gather*}
F_G = G \frac{\rho \frac{4}{3} \pi r^3 M}{R^2} \\
F_S = p A = I / c \cdot \pi r^2 = \frac{P_0}{4 \pi R^2 c} \cdot \pi r^2 = \frac{P_0 r^2}{4 R^2 c} \\
\text{Forholdet mellom de to kreftene blir da} \\
\frac{F_S}{F_G} = \frac{\frac{P_0 r^2}{4 R^2 c}}{G \frac{\rho \frac{4}{3} \pi r^3 M}{R^2}} = \frac{P_0 r^2 3 R^2}{G \rho 4 \pi r^3 M 4 R^2 c} = \frac{3 P_0}{16 G M \rho \pi r c} \\
\text{Hvis kreftene er like blir dette forholdet $1$ og vi kan finne en løsning for r} \\
1 = \frac{3 P_0}{16 G M \rho \pi r c} \Rightarrow r = \frac{3 P_0}{16 G M \rho \pi c} \\
r = \frac{3 \cdot 3.9 \cdot 10^{26} \text{W}}{16 \cdot 6.67 \cdot 10^{-11} \text{Nm}^2/\text{kg}^2 \cdot 1.99 \cdot 10^{30} \text{kg} \cdot 2.5 \cdot 10^{3} \text{kg}/\text{m}^3 \cdot \pi \cdot c} = 2.33 \cdot 10^{-7} \text{m} = 0.233 \mu\text{m}
				\end{gather*}









		\paragraph{10.}
			\subparagraph{12)}
				\begin{flushleft}
Når lyset går fra vann over til glass ved en vinkel på $48.7$ grader vil alt lyset reflekteres. Dette er beskrevet av Snels lov:
$$\sin(\theta) = \frac{n_1}{n_2}$$
der $\theta$ er innfalls- og utfallsvinkelen, $n_1$ er brytningsvinkelen for vann og $n_2$ er brytningsvinkelen for glass. Da kan vi finne brytningsvinkelen ved
$$n_2 = \frac{n_1}{\sin(\theta)} = \frac{1.333}{\sin(48.7 \text{grader})} = 1.774$$
				\end{flushleft}










			\subparagraph{14)}
				\begin{flushleft}
Hvis den reflekterte strålen er fullstendig polarisert må det bety at innfalsvinkelen er Brewster-vinkelen. Det vil si at vi får sammenhengen
$$\tan(\theta) = \frac{n_2}{n_1} = n_2 = \tan(54.5 \text{grader}) = 1.402$$ \\
Vet også at vinkelen til den transmiterte strålen er gitt ved
$$\cos(\theta_t) = \cos(\theta_i) \frac{n_2}{n_1} = \cos(54.5 \text{grader}) 1.402$$
$$\theta_t = \arccos(\cos(54.5 \text{grader}) 1.402) = 35.5 \text{grader}$$
				\end{flushleft}









			\subparagraph{16)}
				\begin{gather*}
\text{Intensiteteten til lysstrålen etter den har passert det andre filteret er gitt} \\
\text{Ved Malus sin lov} \\
I = I_1 \cos^2(\theta_2 - \theta_1) = I_0 \cos^2(-70 \text{grader} - 15 \text{grader}) \\
= 0.087156 I_1 = \frac{0.087156}{2} I_0 = 0.04358 I_0 \\
\text{Hvis vi introduserer et nytt filter med helning $-32$ grader får vi} \\
I_2 = I_1 \cos^2(\theta_2 - \theta_1) = I_0 \cos^2(-32 \text{grader} - 15 \text{grader}) = 0.6820 I_1\\
I = I_2 \cos^2(\theta_3 - \theta_2) = I_0 \cos^2(-70 \text{grader} + 32 \text{grader}) \\
= 0.78801 I_2 = 0.78801 \cdot 0.6820 \cdot 0.5 I_0 = 0.2687 I_0 \\
\text{Hadde vi satt det siste filteret inn på slutten hadde vi istedenfor fått} \\
I = I_2 \cos^2(\theta_2 - \theta_1) = I_0 \cos^2(-32 \text{grader} + 70 \text{grader}) \\
= 0.78801 I_2 = 0.78801 \cdot 0.04358 I_0 = 0.03434 \\
\text{Som vi ser er mye mindre enn det vi fikk med filteret imellom}
				\end{gather*}
\end{document}