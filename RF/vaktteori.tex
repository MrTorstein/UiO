\documentclass[11pt,norsk,a4paper]{article}
\usepackage[utf8]{inputenc}
\usepackage[T1]{fontenc}
\usepackage{babel,graphicx,varioref}
\title{Vaktteori}
\author{Vaktgruppa, Realistforeningen}

\begin{document}
\maketitle

\section{Innledning}

Denne tekstens intensjon er å gi en oversikt over det viktigste vakter bør kjenne til. Erfaring tilsier at man ved å kun drive muntlig og praktisk opplæring har lett for å glemme en del detaljer. Teksten vil derfor gå gjennom vanlige rutiner, og i tillegg forsøke å gå raskest mulig gjennom en del andre temaer det er greit å ha en viss kjennskap til. Det viktigste for nye vakter står i kapittel \vref{rutiner}, rutiner. Som vanlig er tekstens verdi sterkt avhengig av at den blir oppdatert. Kommer du ut for en problematisk situasjon som denne teksten ikke sier noe om, bør du anbefale vaktgruppesjefen å legge til et nytt punkt.

\section{Generelt}
\subsection{Litt historie}

\subsection{Hvordan man behandler folk}

\section{Rutiner under forskjellige typer arrangementer} \label{rutiner}

Det er en del ting vi skal gjøre når vi holder vakt. Det er også en del ting vi \textit{bør} gjøre, og enkelte ting vi \textit{kan} gjøre hvis vi har tid. Det kalles rutiner, og er ganske viktig. Rutinene er litt forskjellige etter hvor stort det aktuelle arrangementet er, så kapittelet er delt inn i forskjellige underpunkter. Rutinene kan enkelt deles inn i før, under og etter arrangementet. Rutiner for store og mellomstore arrangementer er ment som tillegg til det som gjelder små arrangementer.

\subsection{Små arrangementer}

Et \textit{lite} arrangement er typisk en vanlig fredagskveld eller øl og pizza etter en bedriftspresentasjon. Slike kvelder trengs en til to vakter.

\subsubsection{Før}
\begin{itemize}
\item Sjekk glasset i nødåpnerne på dørene i inngangen til VB. Det hender at dette er knust før arrangementet begynner. Da er det greit å si fra til Vaktsentralen før arrangementet begynner, slik at ikke vi får skylda.
\item Sjekk rømningsveier. Sjekk at møbleringen i kjelleren er tilfredsstillende, slik at rømningsveiene internt i kjelleren er åpne. Dette er nærmere beskrevet i kapittel \vref{brann}. Lemp ut møbler som ikke skal være i kjelleren, og sett dem utenfor kjelleren et sted de ikke er i veien ved en eventuell rømning. Dette gjelder selvsagt ikke hvis ekstra møbler trengs til det aktuelle arrangementet. I slike tilfeller må man sørge for å finne andre trygge løsninger. Ofte er vaktgruppesjef orientert på forhånd, og løsning allerede funnet.
\item Kjelleren er ofte rotete. Sjekk at det ikke er tilgjengelig for gjestene ting som ikke bør være tilgjengelig for fulle folk. Dette inkluderer blant annet diverse verktøy, kniver og kjemikalier. Bruk sunn fornuft.
\end{itemize}

\subsubsection{Under}
\begin{itemize}
\item Pass på at folk ikke har med medbragt alkohol. Vi sjekker vanligvis vesker og liknende, men det er ikke det er nødvendig. Følg med på hva folk har på bordene, og spesielt hva de har under bordene. Dette er den vanligste måten vi pleier å oppdage medbragt som har sneket seg inn. Forsøk å være litt diskret, det er en fordel å ikke se ut som om man kontrollerer.
\item Pass på at folk ikke tar med alkohol ut fra VB. Her møter vi et lite problem med måten vi praktiserer det i dag. Vi har vanligvis kun skjenkebevilling for RF-kjelleren, og ikke resten av VB. Det vil si at folk egentlig ikke kan ta med alkohol ut fra kjelleren og opp trappen. Like fullt pleier vi å akseptere øl ut av kjelleren, siden skjenkekontrollen ikke klager på det, men ikke opp trappa.

Vanligvis gjør vi ting litt forskjellig før og etter VBs ordinære stengetid 22:00. Før dette har vi ofte bare en vakt på jobb. Da er det greit for denne vakten å være i kjelleren, men følge med på om det blir tatt med alkohol ut derfra. Det kan godtas at folk tar med alkohol på toalettet, men for øvrig bør vi unngå at folk tar med alkohol rundt i resten av bygget. Det er jo folk som studerer der. I tillegg bør man en gang iblant sjekke utenfor utgangsdøra på VB, enkelte røykere tar med alkohol ut. Etter 22:00 bør vi ha vakt i utgangsdøra fra VB, da passer vi på der om folk tar med alkohol ut. Dette er ikke nødvendigvis lett, og krever en oppmerksom vakt. Men merk at vi med fordel kan begrense mengden alkohol som blir tatt med rundt i resten av VB også. Vi har, som nevnt, strengt tatt bare skjenkebevilling for kjelleren.
\item Sørg for at døra til VB blir holdt åpen etter VBs stengetid klokka 22:00. Ring Vaktsentralen og be om at døra blir holdt åpen. Litt avhengig av hvem som jobber i Vaktsentralen den aktuelle kvelden  kan man bli nødt til å argumentere med at vi skal ha eget vakthold i døra, men det pleier å gå greit. Sørg for at dette blir ordnet \textit{før} 22:00, ellers kommer sannsynligvis noen til å knuse glasset i nødåpneren for å komme ut.
\item Følg med på hvor mange personer det er i kjelleren. Vanligvis har vi lov til å ha 200 personer i kjelleren, men det reduseres til 150 personer hvis vi har foajeefest. å kunne bedømme antall personer krever trening. Det læres lettest ved å spørre erfarne vakter hvor mange de mener det er i kjelleren, og ved å telle selv de kveldene det er få nok gjester der.

Blir det for mange, kan man bli nødt til å improvisere kø utenfor kjelleren, og slippe inn like mange som går ut. Vanligvis gjelder dette større fester enn en ordinær fredag, slik at vi har flere vakter på jobb.
\item Følg med på om gjestene blir for fulle. Dette kan være vanskelig å vurdere, men personer med uvanlig dårlig koordinasjon, usammenhengende snakk eller som spyr er klare eksempler. Personer som gjør generelt tåpelige ting ellers kan ofte også anses som for fulle. Konferer med eventuelle andre vakter i tilfelle tvil. Skjenkemester kan ofte også komme med innspill.

Hva man gjør med personer som er for fulle, varierer litt. Loven er rimelig klar på at de skal kastes ut, og andre personer som sitter ved samme bord som personen som er for full, skal heller ikke ha servering. Personer som er for fulle kan grovt inndeles i to kategorier, de som er bråkete og er vanskelige, og de som ikke er med i det hele tatt. De bråkete kan man enkelt og greit kaste ut. De som ikke er med i det hele tatt er ofte jenter. Disse kan vi ikke uten videre kaste ut, spesielt hvis det er vinter og kaldt. Det er rett og slett ikke trygt. Har de venner i lokalet som kan hjelpe seg hjem, er det ofte en god løsning. Man kan da sørge for at personen som er for full drikker vann eller kaffe frem til vennene er klare til å gå. Personer som er så fulle og ikke har venner i lokalet er heldigvis relativt sjelden. I slike tilfeller kan vi bli nødt til å få hjelp av politiet eller improvisere en annen god løsning.
\end{itemize}



\subsection{Mellomstore arrangementer}

Et \textit{mellomstort} arrangement er typisk en av festene der det rigges spritbar. Fredagspuber der det blir gjort litt ekstra, slik at det kan bli behov for en ekstra vakt, kan også være et mellomstort arrangement. Et eksempel på dette er Oktoberfest. Vanligvis krever mellomstore arrangementer tre til fem vakter.

\subsection{Store arrangementer}

\textit{Store} arrangementer er de festene der vi legger opp til å fylle hele VB. Vanligvis gjelder det semesterstartsfest, pub til pub og Las Vegas i høstsemesteret, og eventuelt enkelte fester under Biørnegildet. Her trengs det fra seks vakter og oppover.

\section{Lov og rett}

Det er ikke helt uten grunn at jus er et eget studium ved universitetet. Lovverket er stort, og til en viss grad vanskelig å tolke. Dette kapittelets intensjon er å gi en oversikt over det viktigste, med noen vanlige tolkninger og et forsøk på å beskrive hva det vil si for vår virksomhet.

\subsection{Hva vi har lov til}

\subsection{Alkoholloven}
Værsågod, Øystein

\section{Førstehjelp}

Medisin er også et eget studium, med god grunn. Dette er ikke ment å være et utfyllende kurs i førstehjelp, til det er førstehjelp et alt for stort tema. Hovedpoenget her er å beskrive hva man bør gjøre i situasjoner det er sannsynlig at vi kan møte på.

\section{Brannvern} \label{brann}

\section{Narkotika}



\section{Tips og triks}

\subsection{Ting fulle folk finner på, og hva man kan gjøre for å fikse det}
\begin{itemize}
 \item {Skrur av vannet til toalettene}

Det har forekommet at morsomme folk har skrudd av vannet til toalettene. Det blir fort ganske kvalmt når det ikke er mulig å skylle ned, så det er greit å få ordnet det. Det er en stengekran i den innerste båsen, ved å åpne denne begynner ting å virke igjen.
\end{itemize}




\end{document}